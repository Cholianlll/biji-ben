\documentclass{article}
\usepackage{fullpage}
\usepackage[normalem]{ulem}
\usepackage{amstext}
\newcommand{\var}[1]{\mathit{#1}}
\setlength{\parskip}{6pt}

\begin{document}

~~~\vspace{-2.0cm}

\noindent
University of Toronto\\
{\sc csc}343, Fall 2015\\[10pt]
{\LARGE\bf Assignment 1: Your name and student number here}

\noindent
Unary operators on relations:
\begin{itemize}
\item $\Pi_{x, y, z} (R)$
\item $\sigma_{condition} (R) $
\item $\rho_{New} (R) $
\item $\rho_{New(a, b, c)} (R) $
\end{itemize}
Binary operators on relations:
\begin{itemize}
\item $R \times S$
\item $R \bowtie S$
\item $R \bowtie_{condition} S$
\item $R \cup S$
\item $R \cap S$
\item $R - S$
\end{itemize}
Logical operators:
\begin{itemize}
\item $\vee$
\item $\wedge$
\item $\neg$
\end{itemize}
Assignment:
\begin{itemize}
\item $New(a, b, c) := R$
\end{itemize}

\noindent
Below is the text of the assignment questions; we suggest you include it in your solution.
We have also included a nonsense example of how a query might look in LaTeX.  
We used \verb|\var| in a couple of places to show what that looks like.  
If you leave it out, most of the time the algebra looks okay, but names
such as ``Offer" look horrific without it.

The characters ``\verb|\\|" create a line break and ``[5pt]" puts in 
five points of extra vertical space.  The algebra is easier to read with extra
vertical space.
We chose ``---" to indicate comments, and added less vertical space between comments
and the algebra they pertain to than between steps in the algebra.
This helps the comments visually stick to the algebra.

%----------------------------------------------------------------------------------------------------------------------
%----------------------------------------------------------------------------------------------------------------------
\section*{Part 1: Queries [84\% - 7 marks each]}


\begin{enumerate}

\item   % ---------- 
Find the last names of the athlete(s) of the country(ies) that did not compete in any event yet.

{\bf Answer}:\\[5pt]
{
Here is a sample of what your answer should like - please replace this 
with the correct answer once you get the hang of using Latex script.\\[5pt]
$
SomeIntermediateRelation := \Pi_{SomeAttribute} (BlahRelation1 \bowtie BlahRelation2) \\[5pt]
OtherIntermediateRelation := \Pi_{OtherAttribute} SomeIntermediateRelation \\[5pt]
Answer := (RandomRelation \bowtie OtherIntermediateRelation)
$
}

\item   % ---------- 
Find the last names of the athlete(s) of the country(ies) that did not win any medals yet (either because they did not compete, or because their athletes did not rank in the top 3 in any event so far).

{\bf Answer}:\\[5pt]
{
$
Answer := 
$
}


\item   % ---------- 
Find the stadium names of all the stadiums where exactly one event took place.

{\bf Answer}:\\[5pt]
{
$
Answer := 
$
}


\item   % ---------- 
Find all the sporting disciplines that Canadian athletes have competed in so far. 

{\bf Answer}:\\[5pt]
{
$
Answer := 
$
}

\item   % ---------- 
Find the first and last name of the athletes whose sporting discipline is ``swimming'' 
and who have won the highest number of gold medals among all athletes who compete 
in the same sport.

{\bf Answer}:\\[5pt]
{
$
Answer := 
$
}


\item   % ---------- 
Find the name of every country that has won at least one of every type of medal 
(gold, silver, and bronze).

{\bf Answer}:\\[5pt]
{
$
Answer := 
$
}


\item   % ---------- 
Find the gold medalist country of the event for which the very first ticket out of 
all the tickets in the database was purchased. A gold medalist country is a country
that has won at least one gold medal. 
 
{\bf Answer}:\\[5pt]
{
$
Answer := 
$
}


\item   % ---------- 
Find the first and last name of the athlete representing ``Mexico'', who so far 
has the second highest number of gold medals (among athletes of the same country).

{\bf Answer}:\\[5pt]
{
$
Answer := 
$
}


\item   % ----------
Find the sports disciplines for events for which at least two tickets 
were bought on the date of the event.

{\bf Answer}:\\[5pt]
{
$
Answer := 
$
}

\item   % ----------
Find the athlete with the highest overall number of gold medals won so far, and report that athlete’s first and last name, country name, and number of gold medals won.

{\bf Answer}:\\[5pt]
{
$
Answer := 
$
}

\item   % ----------
Find the discipline (sport) of the event for which the highest number of tickets was purchased. 

{\bf Answer}:\\[5pt]
{
Answer := 
}

\item   % ---------- 
Find the first and last name for all athletes who have won a gold medal in an event for which no tickets were sold. 

{\bf Answer}:\\[5pt]
{
$
Answer := 
$
}

\end{enumerate}



%----------------------------------------------------------------------------------------------------------------------
\section*{Part 2: Additional Integrity Constraints [16\% - 4 marks each]}

Below are some additional integrity constraints on our schema. Express each of them 
using the notation from Section 2.5 of your textbook. If a constraint cannot be 
expressed using such notations, simply write ``cannot be expressed''.


\begin{enumerate}

\item   % ----------
An athlete cannot win more than one medal type in the same event.

{\bf Answer}:\\[5pt]
{
Example:\\[5pt]
$
R = \emptyset\\[5pt]
$
This is just an exampls, so please replace this \\[5pt]
with the correct answer.
}

\item   % ---------- 
All tickets for an event have to be purchased before the time of the event.

{\bf Answer}:\\[5pt]
{
$
Answer here
$
}


\item   % ---------- 
The number of tickets purchased for an event should not exceed the capacity 
of the stadium where the event takes place.

{\bf Answer}:\\[5pt]
{
$
Answer here.
$
}

\item   % ---------- 
An athlete could not have competed in an event for a sporting discipline 
that they are not qualified to participate in.

{\bf Answer}:\\[5pt]
{
$
Answer here.
$
}

\end{enumerate}
  

\end{document}



