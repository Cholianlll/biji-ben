\documentclass[12pt]{article}

\usepackage{fancyhdr}
\usepackage{extramarks}
\usepackage{amsmath}
\usepackage{amsthm}
\usepackage{amssymb}
\usepackage{amsfonts}
\usepackage{tikz}
\usepackage[plain]{algorithm}
\usepackage{algpseudocode}
\usepackage{graphicx}

\usetikzlibrary{automata,positioning}

%
% Basic Document Settings
%

\topmargin=-0.45in
\evensidemargin=0in
\oddsidemargin=0in
\textwidth=6.5in
\textheight=9.0in
\headsep=0.25in

\linespread{1.1}

\pagestyle{fancy}
\lhead{\hmwkAuthorName}
\chead{\hmwkClass\ \hmwkTitle}
\rhead{\firstxmark}
\lfoot{\lastxmark}
\cfoot{\thepage}

\renewcommand\headrulewidth{0.4pt}
\renewcommand\footrulewidth{0.4pt}

\setlength\parindent{0pt}

%
% Create Problem Sections
%

\newcommand{\enterProblemHeader}[1]{
    \nobreak\extramarks{}{Problem \arabic{#1} continued on next page\ldots}\nobreak{}
    \nobreak\extramarks{Problem \arabic{#1} (continued)}{Problem \arabic{#1} continued on next page\ldots}\nobreak{}
}

\newcommand{\exitProblemHeader}[1]{
    \nobreak\extramarks{Problem \arabic{#1} (continued)}{Problem \arabic{#1} continued on next page\ldots}\nobreak{}
    \stepcounter{#1}
    \nobreak\extramarks{Problem \arabic{#1}}{}\nobreak{}
}

\setcounter{secnumdepth}{0}
\newcounter{partCounter}
\newcounter{homeworkProblemCounter}
\setcounter{homeworkProblemCounter}{1}
\nobreak\extramarks{Problem \arabic{homeworkProblemCounter}}{}\nobreak{}

%
% Homework Problem Environment
%
% This environment takes an optional argument. When given, it will adjust the
% problem counter. This is useful for when the problems given for your
% assignment aren't sequential. See the last 3 problems of this template for an
% example.
%
\newenvironment{homeworkProblem}[1][-1]{
    \ifnum#1>0
        \setcounter{homeworkProblemCounter}{#1}
    \fi
    \section{Problem \arabic{homeworkProblemCounter}}
    \setcounter{partCounter}{1}
    \enterProblemHeader{homeworkProblemCounter}
}{
    \exitProblemHeader{homeworkProblemCounter}
}

%
% Homework Details
%   - Title
%   - Date
%   - Class
%   - Instructor
%   - Author
%

\newcommand{\hmwkTitle}{Tutorial\ \#0}
\newcommand{\hmwkDate}{23 Feb, 2018}
\newcommand{\hmwkClass}{STAT8027}
\newcommand{\hmwkAuthorName}{\textbf{Rui Qiu u6139152}}

%
% Title Page
%

\title{
    \vspace{2in}
    \textmd{\textbf{\hmwkClass:\ \hmwkTitle}}\\
    \normalsize\vspace{0.1in}\small{\hmwkDate}\\
    \vspace{3in}
}

\author{\hmwkAuthorName}
\date{}

\renewcommand{\part}[1]{\textbf{\large Part \Alph{partCounter}}\stepcounter{partCounter}\\}

%
% Various Helper Commands
%

% New QED symbol
\renewcommand{\qedsymbol}{$\blacksquare$}

% Useful for algorithms
\newcommand{\alg}[1]{\textsc{\bfseries \footnotesize #1}}

% For derivatives
\newcommand{\deriv}[1]{\frac{\mathrm{d}}{\mathrm{d}x} (#1)}

% For partial derivatives
\newcommand{\pderiv}[2]{\frac{\partial}{\partial #1} (#2)}

% Integral dx
\newcommand{\dx}{\mathrm{d}x}

% Alias for the Solution section header
\newcommand{\solution}{\textbf{\large Solution}}

% Probability commands: Expectation, Variance, Covariance, Bias
\newcommand{\E}{\mathrm{E}}
\newcommand{\Var}{\mathrm{Var}}
\newcommand{\Cov}{\mathrm{Cov}}
\newcommand{\Bias}{\mathrm{Bias}}

\begin{document}

\maketitle

\pagebreak

\begin{homeworkProblem}[3]

\textbf{(a) Solution:}\\

Since $x>0, y=x^2>0$

\[\begin{split}
F_Y(y)&=P(Y\leq y)=P(x^2\leq y)=P(0\leq x\leq\sqrt{y})=F_x(\sqrt{y})	\\
f_Y(y)&=F'_Y(y)\\
&=F'_X(\sqrt{y})\cdot\frac12 y^{-\frac12}\\
&=f_X(\sqrt{y})\cdot\frac12 y^{-\frac12}\\
&=\sqrt{\frac{2}{\pi}}\exp\left(-\frac12 y\right)\frac12 \frac1{\sqrt{y}}, y>0\\
&=\frac1{\sqrt{2y}\cdot\sqrt{\pi}}\exp\left(-\frac12 y\right)
\end{split}
\]

\textbf{(b) Solution:}\\

As $\Gamma(\frac12)=\sqrt{\pi}$, the PDF of $Y$ can be written as:

\[\frac1{\sqrt{2y}}\frac1{\Gamma(\frac12)}\exp\left(-\frac12 y\right)=\frac{y^{-\frac12}\frac12^{\frac12}e^{-\frac12 y}}{\Gamma(\pi)}\]

So $Y\sim \text{Gamma}(\alpha=\frac12,\lambda=\frac12).$

\end{homeworkProblem}

\begin{homeworkProblem}[4]

\textbf{Proof:}\\

Suppose $A=\{(u_1,u_2): g_1(u_1,u_2)\leq y_1, g_2(u_1,u_2)\leq y_2\}, A_h=\{(v_1,v_2):v_1\leq y_1,v_2\leq y_2\}.$

And by definition, we have

\[\begin{split}
	v_1&=g_1(u_1,u_2), v_2=g_2(u_1,u_2)\\
	u_1&=h_1(v_1,v_2), u_2=h_2(v_1,v_2)
\end{split}\]

The CDF of joint distribution of $Y_1,Y_2$ can be written as:

\[
\begin{split}
	F_{Y_1Y_2}(y_1,y_2)=P(A_h)&=P(u_1\leq h_1(y_1,y_2),u_2\leq h_2(y_1,y_2))\\
	&=F_{X_1X_2}(h_1(y_1,y_2),h_2(y_1,y_2))\\
	&=\int\int_A f_{X_1X_2}(u_1,u_2)du_1du_2\\
	&=\int\int_{A_h}f_{X_1X_2}(h_1(v_1,v_2),h_2(v_1,v_2))\mid J(v_1,v_2)\mid dv_1dv_2
\end{split}
\]

Therefore, the PDF of joint distribution of $Y_1,Y_2$

\[
\begin{split}
	f_{Y_1Y_2}(y_1,y_2)&=f_{X_1X_2}(h_1(y_1,y_2),h_2(y_1,y_2))\mid J(y_1,y_2)\mid\\
	&\text{where } J(y_1,y_2) \text{is the Jacobian matrix.}
\end{split}
\]

\qed

\end{homeworkProblem}

\end{document}
