\documentclass[a4paper, 11pt, twoside]{article}
\usepackage{amssymb}
\usepackage{amsmath}
\begin{document}
\title{STAT6038 week 5 lecture 13}
\author{Rui Qiu}
\date{2017-03-22}

\maketitle

blah, blah, blah.

\paragraph{Confidence and Prediction Intervals}\ \\

For new values of $Y$ given new values of $X$

\[\hat{Y}|(X=x^*) = \hat{\beta_0} +\hat{\beta_1}x^*\]

A $95\%$ \textbf{confidence interval} for $E(Y|X=x^*)$ is

\[\hat{Y}\pm t_{\text{error df}}(0.975)\cdot s\sqrt{\frac1n + \frac{(x^*-\bar{x})^2}{s_{xx}}}.\]

Note 

\begin{itemize}
	\item when $x^*=0$, standard error becomes $se(\hat{\beta_0}).$
	\item when $x^*=\bar{x}$, standard error becomes $se(\bar{y})=\frac{s}{\sqrt{n}}.$\\
\end{itemize}

A $95\%$ \textbf{prediction interval} for $Y|X = x^*$ is

\[\hat{Y}\pm t_{\text{error df}}(0.975) \cdot s\sqrt{1+\frac{1}{n}+\frac{(x^*-\bar{x})^2}{s_{xx}}}.\]

Note these are the formulae for SLR, we need to make the usual modifications (switch to matrix notation) for multiple regression.\\

\paragraph{Modelling process} Propose an initial plausible model.



\begin{itemize}
	\item Is the model appropriate? (Are the underlying assumptions ok?)\\
	$\rightarrow$ Errors $\overset{iid}\sim N(0, \sigma^2)$\\
	we estimate these Errors using the residuals \& produce residual plots. (plot(model) in R)
	\item Is the model adequate? (Does the model have significant explanatory power?)\\
	Is it a useful model (as per George Box)\\
	$\rightarrow$ overall F test from the anova table is a good start here: anova(model) in R.
	\item If yes to both the above, then we look at the details of the model by try and answer the research question\\
	$\rightarrow$ this involves looking at the estimated model coefficients: summary(model) in R.
	\item Finally, maybe, if everything is good enough, also use predict(model) in R.
	\item Finally, some overall assessment - is the model just exploratory or can it be sensibly used to make predictions? (i.e., predictive model)\\
	$\rightarrow$ this is a matter of judgement based on an objective assessment of all the above.
\end{itemize}

\end{document}