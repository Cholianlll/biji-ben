\documentclass[a4paper, 11pt, twoside]{article}
\usepackage{amssymb}
\usepackage{amsmath}
\begin{document}
\title{STAT6038 week 6 lecture 18}
\author{Rui Qiu}
\date{2017-03-31}

\maketitle

\paragraph{Summary} So far, what have we learned? The general procedure for examining the relationship between two variables proceeds as follows:

\begin{enumerate}
	\item \textbf{Start with a little exploratory data analysis.} Examine a plot of the data to see if a linear association seems a plausible explanation of the scatterplot.
	\item Using scientific or other background information, transform the data appropriately.
	\item Fit a linear regression model to the two variables using least-square estimates.
	\item Use residual plots and normal q-q plots to examine the plausibility of the basic assumptions of the model.
	\item If necessary (based on the plots of previous step) transform the data again (or move to a more complicated model structure).\\
	\textbf{Note:} We cycle through step 3,4,5 until we are happy with the residual plots.
	\item Re-fit the regression and again examine the residuals thoroughly.\\
	\textbf{Note:} After 5, 6 we get an ``appropriate`` model, do step 7 and look at the first ANOVA table, if F test okay then it is an ``adequate`` model, then we look at the summary output.
	\item Test the significance of the regression and make required predictions.
	\item When satisfactory residual analyses have been reached, re-test and re-predict as required (remembering to transform back to appropriate scales if necessary).\\
\end{enumerate}

We will now take up the point mentioned parenthetically in step 5, and examine more complicated, though still linear models, where we will allow for the possibility of more than one predictor variable.

\end{document}