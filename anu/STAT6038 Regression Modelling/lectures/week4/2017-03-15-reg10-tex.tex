\documentclass[a4paper, 11pt, twoside]{article}
\usepackage{amssymb}
\usepackage{amsmath}
\begin{document}
\title{STAT6038 week 4 lecture 10}
\author{Rui Qiu}
\date{2017-03-15}

\maketitle

\textbf{The overall F-test} (in the ANOVA table for SLR \& also the last line in the summary output) and the \textbf{t-test} on the slope coefficient (for gestation, the $X$ variable) are equivalent for simple linear regression (SLR) and they both answer the question

\[\text{"Is $Y$ (protein) related to $X$ (gestation)?"}\]

They are not the only inferences we can make using the summary output.\\

Say the (a priori) question had been "do the protein levels increase by more than $0.02$ mg/mL for each of gestation?

\begin{itemize}
	\item Step I: $H_0: \beta_1 = 0.02, H_A : \beta_1 > 0.02$
	\item Step II: $t=\frac{\hat{\beta_1}-\beta_1|H_0}{se(\hat{\beta_1}}$\\
	$se(\hat{\beta_1})$ these are still the values given in the summary output.
	\item Step III: $\alpha=0.05$, reject $H_0$ if observed $t > t_{17}(0.95)$, use code qt(...).
	\item Step IV: Put $0.95$ in the lower tail, $\alpha=0.05$ to right. so $t_{17}(0.95)=1.79$ (theoretical t-statistics), but the observed t-statistics is $0.86$, with $p$-value $0.2$.\\
	So \textbf{not reject $H_0$}.
	\item Step V: As observed $t=0.086 \not > 1.74 = t_{17}(0.95)$\\
	OR\\
	as $p=0.20 > \alpha = 0.05$\\
	DO NOT REJECT $H_0$ \& conclude that the expected increase in protein levels is of the order of $0.02$ mg/mL for each additional week of gestation, but is NOT significantly greater than that. 
\end{itemize}

\paragraph{T-test on the intercept coefficient}\ \\
\begin{itemize}
	\item Step I: $H_0:\beta_0 = 0$ vs $H_A:\beta_0\not=0$
	\item Step II: $t=\frac{\hat{\beta_0}-0}{se(\hat{\beta_0}}\sim t_{n-2}$\\
	where $se$ is $s\sqrt{\frac1{n}+\frac{\bar{x}^2}{s_{xx}}}$
	\item Step III: $\alpha=0.05$; reject $H_0$ if observed $t < t_{17}(0.025)$ or observed $t > t_{17}(0.975)$.
	\item Step IV: two tail t-test. two sides each have a tail with $\frac{\alpha}{2}=0.025$\\
	In fact, we have observed $t=2.42 > 2.11=t_{17}(0.025)$\\
	What about p-value approach? p-value is $0.027$.
	\item Step V: As $p=0.027 < \alpha = 0.05$\\
	reject $H_0$ and conclude $H_A: \beta_0\not=0.$
\end{itemize}

The plot shows that $\hat{\beta_0}$ (the intercept) is not zero.\\

But the (possible) true relationship could be exponential(?), as it passes through our \textbf{range of data} as well.

So \textbf{regression models are at test a good "local linear approximation".}

\end{document}