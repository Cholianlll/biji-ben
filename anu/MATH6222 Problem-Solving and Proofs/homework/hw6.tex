\documentclass[12pt]{article}

\usepackage{fancyhdr}
\usepackage{extramarks}
\usepackage{amsmath}
\usepackage{amsthm}
\usepackage{amssymb}
\usepackage{amsfonts}
\usepackage{tikz}
\usepackage[plain]{algorithm}
\usepackage{algpseudocode}
\usepackage{graphicx}

\usetikzlibrary{automata,positioning}

%
% Basic Document Settings
%

\topmargin=-0.45in
\evensidemargin=0in
\oddsidemargin=0in
\textwidth=6.5in
\textheight=9.0in
\headsep=0.25in

\linespread{1.1}

\pagestyle{fancy}
\lhead{\hmwkAuthorName}
\chead{\hmwkClass\ \hmwkTitle}
\rhead{\firstxmark}
\lfoot{\lastxmark}
\cfoot{\thepage}

\renewcommand\headrulewidth{0.4pt}
\renewcommand\footrulewidth{0.4pt}

\setlength\parindent{0pt}

%
% Create Problem Sections
%

\newcommand{\enterProblemHeader}[1]{
    \nobreak\extramarks{}{Problem \arabic{#1} continued on next page\ldots}\nobreak{}
    \nobreak\extramarks{Problem \arabic{#1} (continued)}{Problem \arabic{#1} continued on next page\ldots}\nobreak{}
}

\newcommand{\exitProblemHeader}[1]{
    \nobreak\extramarks{Problem \arabic{#1} (continued)}{Problem \arabic{#1} continued on next page\ldots}\nobreak{}
    \stepcounter{#1}
    \nobreak\extramarks{Problem \arabic{#1}}{}\nobreak{}
}

\setcounter{secnumdepth}{0}
\newcounter{partCounter}
\newcounter{homeworkProblemCounter}
\setcounter{homeworkProblemCounter}{1}
\nobreak\extramarks{Problem \arabic{homeworkProblemCounter}}{}\nobreak{}

%
% Homework Problem Environment
%
% This environment takes an optional argument. When given, it will adjust the
% problem counter. This is useful for when the problems given for your
% assignment aren't sequential. See the last 3 problems of this template for an
% example.
%
\newenvironment{homeworkProblem}[1][-1]{
    \ifnum#1>0
        \setcounter{homeworkProblemCounter}{#1}
    \fi
    \section{Problem \arabic{homeworkProblemCounter}}
    \setcounter{partCounter}{1}
    \enterProblemHeader{homeworkProblemCounter}
}{
    \exitProblemHeader{homeworkProblemCounter}
}

%
% Homework Details
%   - Title
%   - Date
%   - Class
%   - Instructor
%   - Author
%

\newcommand{\hmwkTitle}{Homework\ \#6}
\newcommand{\hmwkDate}{2017-04-04}
\newcommand{\hmwkClass}{MATH6222}
\newcommand{\hmwkClassInstructor}{Instructor: Dr. David Smyth}
\newcommand{\hmwkTutor}{Tutor: Mark Bugden (Wednesday 1-2pm)}
\newcommand{\hmwkAuthorName}{\textbf{Rui Qiu u6139152}}

%
% Title Page
%

\title{
    \vspace{2in}
    \textmd{\textbf{\hmwkClass:\ \hmwkTitle}}\\
    \normalsize\vspace{0.1in}\small{\hmwkDate}\\
    \vspace{0.1in}\large{\textit{\hmwkClassInstructor}}\\
    \vspace{0.1in}
    	\large{\textit{\hmwkTutor}}
    \vspace{3in}
}

\author{\hmwkAuthorName}
\date{}

\renewcommand{\part}[1]{\textbf{\large Part \Alph{partCounter}}\stepcounter{partCounter}\\}

%
% Various Helper Commands
%

% New QED symbol
\renewcommand{\qedsymbol}{$\blacksquare$}

% Useful for algorithms
\newcommand{\alg}[1]{\textsc{\bfseries \footnotesize #1}}

% For derivatives
\newcommand{\deriv}[1]{\frac{\mathrm{d}}{\mathrm{d}x} (#1)}

% For partial derivatives
\newcommand{\pderiv}[2]{\frac{\partial}{\partial #1} (#2)}

% Integral dx
\newcommand{\dx}{\mathrm{d}x}

% Alias for the Solution section header
\newcommand{\solution}{\textbf{\large Solution}}

% Probability commands: Expectation, Variance, Covariance, Bias
\newcommand{\E}{\mathrm{E}}
\newcommand{\Var}{\mathrm{Var}}
\newcommand{\Cov}{\mathrm{Cov}}
\newcommand{\Bias}{\mathrm{Bias}}

\begin{document}

\maketitle

\pagebreak

\begin{homeworkProblem}
Let $a,b\in\mathbb{Z}$,

(a) Prove that $\gcd(a+b, a-b)=gcd(2a, a-b)=gcd(a+b,2b).$

(b) Suppose that $\gcd(a,b)=1$. What can you say about $\gcd(a^2, b^2)$? What about $\gcd(a,2b)$?\\

\textbf{(a) Proof:} Suppose $d\in\mathbb{Z}, d|(a+b), d|(a-b)$, then we claim that $d$ divides the sum and difference of such two integers:

\[
\begin{split}
	d|(a+b+a-b)&\implies d|(2a)\\
	d|(a+b-a+b)&\implies d|(2b)
\end{split}
\]

The reasoning is, suppose $\exists\ k, j \in\mathbb{Z}, a+b=dk, a-b=dj$, then

\[
\begin{split}
	2a &= a+b+a-b = d(k+j)\\
	2b &= a+b-a+b = d(k-j)	
\end{split}
\]

For the same reasoning, when we have $d|(2a), d|(a-b)$, then automatically we have

\[
\begin{split}
	d|(2a-a+b) &\implies d|(a+b)\\
	d|(a+b-a+b) &\implies d|(2b)
\end{split}
\]

Similarly, when we have $d|(a+b), d|(2b)$, we have the following at the same time:

\[
\begin{split}
	d|(a+b-2b) &\implies d|(a-b)\\
	d|(a+b+a-b) &\implies d|(2a)
\end{split}
\]

To conclude, for the 3 pairs of integers, if we have a common divisor $d$ for one of the pairs, then it is automatically a common divisor of the other two pairs. This is equivalent to say, the set of common divisors of the 3 pairs are the same. So the \textbf{greatest} common divisors of the 3 pairs are the same.

\qed
\\

\textbf{(b) Proof:}
Suppose $a = p_1^{k_1}p_2^{k_2}\cdots p_i^{k_i}, b=q_1^{l_1}q_2^{l_2}\cdots q_j^{l_j},$ where all $p$'s and $q$'s are prime factors (reordered with $p_1<p_2<\cdots <p_i$ and $q_1<q_2<\cdots < q_j$ for simplicity), and $k$'s and $l$'s are integers.\\

Since $\gcd(a,b)=1$, $p_s\not=q_t$ for any integers $1<s<i, 1<t<j$.\\

Simply squaring $a, b$ will only double the exponents of prime factorizations of $a$ and $b$, i.e.

\[
\begin{split}
	a^2 &= p_1^{2k_1}p_2^{2k_2}\cdots p_i^{2k_i}\\
	b^2 &=q_1^{2l_1}q_2^{2l_2}\cdots q_j^{2l_j}
\end{split}
\]

Still, $p_s\not=q_t$ for any prime factors of $a$ and $b$, no more common factors added. The greatest common divisor of $a$ and $b$ remains the same, i.e. $\gcd(a^2,b^2)=\gcd(a,b)=1$.\\

However, when it comes to $\gcd(a, 2b)$, there are two possible cases:

\begin{itemize}
	\item If $a$ has no prime factor $2$, then $2b=2q_1^{l_1}q_2^{l_2}\cdots q_j^{l_j}$ has an extra prime factor $2$ which is still a ``common" factor, so the greatest common divisor won't change.
	\item If $a$ has prime factor $2$ already as $a = 2p_2^{k_2}\cdots p_i^{k_i}$, then $2b$ will give our pair a new common factor $2$, i.e. $\gcd(a, 2b)=2 \not=\gcd(a, b)=1.$
\end{itemize}

To conclude, $\gcd(a^2,b^2)=1$ but $\gcd(a, 2b)=1$ or $2$.

\qed

\end{homeworkProblem}

\begin{homeworkProblem}[3]
Show that the gaps between primes can be arbitrarily large. Do this by constructing, for any positive integer $n$, a set of $n$ consecutive integers that are not prime. (Hint: Determine a positive integer $x$ such that $x$ is divisible by $2, x+1$ is divisible by $3, x+2$ is divisible by $4$, etc.)\\

\textbf{Proof:} Suppose such set $S$ contains consecutive non-prime integers starting from $x$ with cardinality $n$.

\[S=\{x, x+1, x+2, \dots, x+n-1\}\]

Also according to the hint, we would like to have $2|x, 3|(x+1), 4|(x+2), \dots, (n+1)|(x+n-1)$. \\

Since $(n+1)!=1\cdot 2\cdots (n+1)$, $(n+1)!$ is divisible by $n$ consecutive integers from $2$ to $n+1$. If $x$ equals to $(n+1)!$, we obviously have $2|x$, but we are not sure about $3|((n+1)!+1)$.\\

How are we gonna fix this? We know that $(n+1)!+3 = 3(2\cdot 4\cdot 5\cdot 6\cdots (n+1)+1)$. And similarly, $(n+1)!+4 = 4(2\cdot 3\cdot 5\cdot 6\cdots (n+1)+1)$. \\

Therefore, we define $x=(n+1)!+2$ instead of just $(n+1)!$, then we will have $n$ consecutive non-prime integers, namely, $x, x+1, \dots, x+n-1$, which are divisible by $2,3,4,\dots, n-1$ correspondingly.\\

Hence, for integer $n$, if we make $n$ arbitrarily large, then there are always $n$ number of consecutive integers that are not prime, i.e. the gap between primes is arbitrarily large.

\qed

\end{homeworkProblem}

\begin{homeworkProblem}
Let $p$ be a prime number.

(a) Prove that $p$ divides ${p\choose k}$ for any $1\leq k\leq p-1.$

(b) Prove that $n^p-n$ is divisible by $p$ for every $n\in\mathbb{N}$. (Hint: Use the binomial theorem and part (a) in a proof by induction.)\\

\textbf{(a) Proof:} We can expand ${p\choose k}$:

\[{p\choose k} = \frac{p!}{k!(p-k)!}=\frac{(p-k+1)\cdot (p-k+2)\cdots p}{1\cdot 2\cdot 3\cdots k}\]

Since $p$ is a prime number, which is only divisible by $1$ and itself, it cannot be canceled out by any integers in the interval $1\leq k \leq p-1 < p$ in the denominator. So ${p\choose k}=p\cdot K$, where $K= \frac{(p-k+1)\cdot (p-k+2)\cdots (p-1)}{1\cdot 2\cdot 3\cdots k}$, i.e. $p|{p\choose k}$.

\qed\\

\textbf{(b) Proof:} Proof by induction on $n$.\\

Base step: $n=1, n^p-n=1^p-1=0$ for prime $p$. And by convention, $0$ is divisible by $p$.\\

Inductive hypothesis: Suppose $n=k$, we claim that $k^p-k$ is divisible by prime $p$.\\

We want to show for $n=k+1$, $(k+1)^p-(k+1)$ is divisible by prime $p$ as well.

Recall the Binomial Theorem which states that:

\[(a+b)^n=\sum\limits^n_{i=0}{n\choose i}a^{n-i}b^i.\]

Specifically,

\[(1+x)^n=\sum\limits^n_{i=0}{n\choose i}x^i.\]

where $a, b$ are integers. And we apply it on our desired formula:

\[
\begin{split}
	(k+1)^p-(k+1) &= \sum\limits^p_{i=0}{p\choose i}k^{i}\cdot 1^{p-i}-(k+1)\\
	&=\sum\limits^p_{i=0}{p\choose i}k^i - (k+1)\\
	&=\sum\limits^{p-1}_{i=0}{p\choose i}k^i+{p\choose p}k^p-k-1\\
	&={p\choose 0}k^0+\sum\limits^{p-1}_{i=1}{p\choose i}k^i +(k^p-k) - 1\\
	&=\sum\limits^{p-1}_{i=1}{p\choose i}k^i + (k^p-k)
\end{split}
\]

By part (a), ${p \choose i}$ is divisible by $p$ for any $1\leq i\leq p-1$, so the sum of ${p\choose i}$ is also divisible by $p$.\\

By inductive hypothesis, $k^p-k$ is divisible by $p$. \\

Therefore, as the sum of sum of combinations ${p\choose i}$ and $k^p-k$, $(k+1)^p-(k+1)$ is also divisible by $p$, i.e. we've proved the case for $n=k+1$.\\

Hence $n^p-n$ is divisible by $p$ for every $n\in\mathbb{N}.$

\qed

\end{homeworkProblem}

%
% Non sequential homework problems
%

% Jump
%\begin{homeworkProblem}[5]

%\end{homeworkProblem}

\end{document}
