\documentclass[12pt]{article}

\usepackage{fancyhdr}
\usepackage{extramarks}
\usepackage{amsmath}
\usepackage{amsthm}
\usepackage{amssymb}
\usepackage{amsfonts}
\usepackage{tikz}
\usepackage[plain]{algorithm}
\usepackage{algpseudocode}
\usepackage{graphicx}

\usetikzlibrary{automata,positioning}

%
% Basic Document Settings
%

\topmargin=-0.45in
\evensidemargin=0in
\oddsidemargin=0in
\textwidth=6.5in
\textheight=9.0in
\headsep=0.25in

\linespread{1.1}

\pagestyle{fancy}
\lhead{\hmwkAuthorName}
\chead{\hmwkClass\ \hmwkTitle}
\rhead{\firstxmark}
\lfoot{\lastxmark}
\cfoot{\thepage}

\renewcommand\headrulewidth{0.4pt}
\renewcommand\footrulewidth{0.4pt}

\setlength\parindent{0pt}

%
% Create Problem Sections
%

\newcommand{\enterProblemHeader}[1]{
    \nobreak\extramarks{}{Problem \arabic{#1} continued on next page\ldots}\nobreak{}
    \nobreak\extramarks{Problem \arabic{#1} (continued)}{Problem \arabic{#1} continued on next page\ldots}\nobreak{}
}

\newcommand{\exitProblemHeader}[1]{
    \nobreak\extramarks{Problem \arabic{#1} (continued)}{Problem \arabic{#1} continued on next page\ldots}\nobreak{}
    \stepcounter{#1}
    \nobreak\extramarks{Problem \arabic{#1}}{}\nobreak{}
}

\setcounter{secnumdepth}{0}
\newcounter{partCounter}
\newcounter{homeworkProblemCounter}
\setcounter{homeworkProblemCounter}{1}
\nobreak\extramarks{Problem \arabic{homeworkProblemCounter}}{}\nobreak{}

%
% Homework Problem Environment
%
% This environment takes an optional argument. When given, it will adjust the
% problem counter. This is useful for when the problems given for your
% assignment aren't sequential. See the last 3 problems of this template for an
% example.
%
\newenvironment{homeworkProblem}[1][-1]{
    \ifnum#1>0
        \setcounter{homeworkProblemCounter}{#1}
    \fi
    \section{Problem \arabic{homeworkProblemCounter}}
    \setcounter{partCounter}{1}
    \enterProblemHeader{homeworkProblemCounter}
}{
    \exitProblemHeader{homeworkProblemCounter}
}

%
% Homework Details
%   - Title
%   - Date
%   - Class
%   - Instructor
%   - Author
%

\newcommand{\hmwkTitle}{Homework\ \#7}
\newcommand{\hmwkDate}{2017-04-24}
\newcommand{\hmwkClass}{MATH6222}
\newcommand{\hmwkClassInstructor}{Instructor: Dr. David Smyth}
\newcommand{\hmwkTutor}{Tutor: Mark Bugden (Wednesday 1-2pm)}
\newcommand{\hmwkAuthorName}{\textbf{Rui Qiu u6139152}}

%
% Title Page
%

\title{
    \vspace{2in}
    \textmd{\textbf{\hmwkClass:\ \hmwkTitle}}\\
    \normalsize\vspace{0.1in}\small{\hmwkDate}\\
    \vspace{0.1in}\large{\textit{\hmwkClassInstructor}}\\
    \vspace{0.1in}
    	\large{\textit{\hmwkTutor}}
    \vspace{3in}
}

\author{\hmwkAuthorName}
\date{}

\renewcommand{\part}[1]{\textbf{\large Part \Alph{partCounter}}\stepcounter{partCounter}\\}

%
% Various Helper Commands
%

% New QED symbol
\renewcommand{\qedsymbol}{$\blacksquare$}

% Useful for algorithms
\newcommand{\alg}[1]{\textsc{\bfseries \footnotesize #1}}

% For derivatives
\newcommand{\deriv}[1]{\frac{\mathrm{d}}{\mathrm{d}x} (#1)}

% For partial derivatives
\newcommand{\pderiv}[2]{\frac{\partial}{\partial #1} (#2)}

% Integral dx
\newcommand{\dx}{\mathrm{d}x}

% Alias for the Solution section header
\newcommand{\solution}{\textbf{\large Solution}}

% Probability commands: Expectation, Variance, Covariance, Bias
\newcommand{\E}{\mathrm{E}}
\newcommand{\Var}{\mathrm{Var}}
\newcommand{\Cov}{\mathrm{Cov}}
\newcommand{\Bias}{\mathrm{Bias}}

\begin{document}

\maketitle

\pagebreak

\begin{homeworkProblem}
(a) Reduce $2^{100} \mod 13$.

\textbf{Solution:}

\[
\begin{split}
	2^{100}\mod 13&=(2^4)^{25}\mod 13\\
	&\equiv3^{25}\mod 13\\
	&\equiv (3^3)^8\cdot 3 \mod 13\\
	&\equiv 1^8\cdot3\mod 13\\
	&\equiv 3\mod 13
\end{split}
\]

(b) Reduce $11^{1000}\mod 8$.

\textbf{Solution:}

\[
\begin{split}
	11^{1000}\mod 8 &\equiv 121^{500}\mod 8\\
	&\equiv 1^{500}\mod 8\\
	&\equiv 1\mod 8
\end{split}
\]
\end{homeworkProblem}

\begin{homeworkProblem}
Let $a,b,c\in\mathbb{Z}$, and suppose that $5$ divides $a^2+b^2+c^2$. Prove that $5$ divides at least one of $a,b,$ or $c$.\\

\textbf{Proof:} As we know from the problem $a^2+b^2+c^2\equiv 0 \mod 5$. Then we can prove by contradiction.

Suppose $5$ divides none of the $a, b, c$, i.e. $a\not\equiv 0\mod 5$, in fact $a$ is in one of the congruence classes $\overline{1},\overline{2},\overline{3},\overline{4}$. Similarly for $b$ and $c$.

Then $a^2$ is among the congruence classes $\overline{1}, \overline{4}, \overline{9}, \overline{16}$, i.e. either $\overline{1}, \overline{4}$. Similarly, for $b^2$ and $c^2$.

Then the sum $a^2+b^2+c^2$ is of congruence classes:

\[
\begin{split}
	1+1+1&\equiv 3\mod 5\\
	1+1+4&\equiv 1\mod 5\\
	1+4+1&\equiv 1\mod 5\\
	1+4+4&\equiv 4\mod 5\\
	4+1+1&\equiv 1\mod 5\\
	4+1+4&\equiv 4\mod 5\\
	4+4+1&\equiv 4\mod 5\\
	4+4+4&\equiv 2\mod 5\\
\end{split}
\]

So none of it is in $\overline{0}$, contradicting the fact that $5|(a^2+b^2+c^2)$.

Hence $5$ divides at least one of $a,b,$ or $c$.

\qed
\end{homeworkProblem}

\begin{homeworkProblem}
Prove that every year (including leap years) has at least one Friday the 13th. What is the maximum number of Friday the 13ths in a year?\\

\textbf{Proof:} First we claim that a month has a Friday the 13th if and only if it begins with a Sunday. The proof of this claim is very direct, we can write down every day from the 1st to the 13th.

Now we index every day from Sunday to Saturday with integers $0$ to $6$, i.e. Sunday is represented by number $0$. Then we will have the following:

\begin{itemize}
	\item January begins on day $x \mod 7$,
	\item February begins on day $x+31 \equiv x+3 \mod 7$,
	\item March begins on day $x+3+28 \equiv x+3\mod 7$,
	\item April begins on day $x+3+31 \equiv x+6\mod 7$,
	\item May begins on day $x+6+30 \equiv x+1\mod 7$,
	\item June begins on day $x+1+31 \equiv x+4\mod 7$,
	\item July begins on day $x+4+30 \equiv x+6\mod 7$,
	\item August begins on day $x+6+31 \equiv x+2\mod 7$,
	\item September begins on day $x+2+31\equiv x+5\mod 7$.
\end{itemize}

So far, up to September, we already have all congruence classes of modulo $7$. That means, no matter what day this year starts on, i.e. no matter what value $x$ is (from $0$ to $6$), there must be at least one value that is of modulo $0$ congruence class of $7$. So that month starts on a Sunday, therefore, it contains a Friday the 13th. 

But we continue our process for the second question:

\begin{itemize}
	\item October begins on day $x+5+30\equiv x\mod 7$,
	\item November begins on day $x+31\equiv x+3\mod 7$,
	\item December begins on day $x+3+30\equiv x+5\mod 7$.
\end{itemize}

Similarly, for a leap year:

\begin{itemize}
	\item January begins on day $x \mod 7$,
	\item February begins on day $x+31 \equiv x+3\mod 7$,
	\item March begins on day $x+3+29 \equiv x+4\mod 7$,
	\item April begins on day $x+4+31\equiv x\mod 7$,
	\item May begins on day $x+30\equiv x+2\mod 7$,
	\item June begins on day $x+2+31\equiv x+5\mod 7$,
	\item July begins on day $x+5+30\equiv x\mod 7$,
	\item August begins on day $x+31\equiv x+3\mod 7$,
	\item September begins on day $x+3+31\equiv x+6\mod 7$,
	\item October begins on day $x+6+30\equiv x+1\mod 7$,
	\item November begins on day $x+1+31\equiv x+4\mod 7$,
	\item December begins on day $x+4+30\equiv x+ 6\mod 7$.
\end{itemize}

As we can see, the starting days of a leap year also every congruence classes of modulo $7$. So at least one starts with Sunday, therefore, at least we have a Friday the 13th.\\

For the question about maximum number of Friday the 13ths in a year,

\begin{itemize}
	\item In a non-leap year, $x+3\mod 7$ appears 3 times, which is the most frequent congruence class. So we set it to be $0\mod 7$, i.e. $x=4$, that year begins with a January the 1st on Thurday.
	\item In a leap year, $x\mod 7 $ appears 3 times, which is the most in this case. So we set $x=0$, i.e. the first day of that year is a Sunday.
\end{itemize}

Hence, generally speaking, the maximum number of Friday the 13ths in a year is $3$.

\qed
\end{homeworkProblem}

%
% Non sequential homework problems
%

% Jump
%\begin{homeworkProblem}[5]

%\end{homeworkProblem}

\end{document}
