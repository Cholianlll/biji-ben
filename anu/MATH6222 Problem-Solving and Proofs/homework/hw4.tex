\documentclass[12pt]{article}

\usepackage{fancyhdr}
\usepackage{extramarks}
\usepackage{amsmath}
\usepackage{amsthm}
\usepackage{amssymb}
\usepackage{amsfonts}
\usepackage{tikz}
\usepackage[plain]{algorithm}
\usepackage{algpseudocode}
\usepackage{graphicx}

\usetikzlibrary{automata,positioning}

%
% Basic Document Settings
%

\topmargin=-0.45in
\evensidemargin=0in
\oddsidemargin=0in
\textwidth=6.5in
\textheight=9.0in
\headsep=0.25in

\linespread{1.1}

\pagestyle{fancy}
\lhead{\hmwkAuthorName}
\chead{\hmwkClass\ \hmwkTitle}
\rhead{\firstxmark}
\lfoot{\lastxmark}
\cfoot{\thepage}

\renewcommand\headrulewidth{0.4pt}
\renewcommand\footrulewidth{0.4pt}

\setlength\parindent{0pt}

%
% Create Problem Sections
%

\newcommand{\enterProblemHeader}[1]{
    \nobreak\extramarks{}{Problem \arabic{#1} continued on next page\ldots}\nobreak{}
    \nobreak\extramarks{Problem \arabic{#1} (continued)}{Problem \arabic{#1} continued on next page\ldots}\nobreak{}
}

\newcommand{\exitProblemHeader}[1]{
    \nobreak\extramarks{Problem \arabic{#1} (continued)}{Problem \arabic{#1} continued on next page\ldots}\nobreak{}
    \stepcounter{#1}
    \nobreak\extramarks{Problem \arabic{#1}}{}\nobreak{}
}

\setcounter{secnumdepth}{0}
\newcounter{partCounter}
\newcounter{homeworkProblemCounter}
\setcounter{homeworkProblemCounter}{1}
\nobreak\extramarks{Problem \arabic{homeworkProblemCounter}}{}\nobreak{}

%
% Homework Problem Environment
%
% This environment takes an optional argument. When given, it will adjust the
% problem counter. This is useful for when the problems given for your
% assignment aren't sequential. See the last 3 problems of this template for an
% example.
%
\newenvironment{homeworkProblem}[1][-1]{
    \ifnum#1>0
        \setcounter{homeworkProblemCounter}{#1}
    \fi
    \section{Problem \arabic{homeworkProblemCounter}}
    \setcounter{partCounter}{1}
    \enterProblemHeader{homeworkProblemCounter}
}{
    \exitProblemHeader{homeworkProblemCounter}
}

%
% Homework Details
%   - Title
%   - Date
%   - Class
%   - Instructor
%   - Author
%

\newcommand{\hmwkTitle}{Homework\ \#4}
\newcommand{\hmwkDate}{2017-03-17}
\newcommand{\hmwkClass}{MATH6222}
\newcommand{\hmwkClassInstructor}{Instructor: Dr. David Smyth}
\newcommand{\hmwkTutor}{Tutor: Mark Bugden (Wednesday 1-2pm)}
\newcommand{\hmwkAuthorName}{\textbf{Rui Qiu u6139152}}

%
% Title Page
%

\title{
    \vspace{2in}
    \textmd{\textbf{\hmwkClass:\ \hmwkTitle}}\\
    \normalsize\vspace{0.1in}\small{\hmwkDate}\\
    \vspace{0.1in}\large{\textit{\hmwkClassInstructor}}\\
    \vspace{0.1in}
    	\large{\textit{\hmwkTutor}}
    \vspace{3in}
}

\author{\hmwkAuthorName}
\date{}

\renewcommand{\part}[1]{\textbf{\large Part \Alph{partCounter}}\stepcounter{partCounter}\\}

%
% Various Helper Commands
%

% New QED symbol
\renewcommand{\qedsymbol}{$\blacksquare$}

% Useful for algorithms
\newcommand{\alg}[1]{\textsc{\bfseries \footnotesize #1}}

% For derivatives
\newcommand{\deriv}[1]{\frac{\mathrm{d}}{\mathrm{d}x} (#1)}

% For partial derivatives
\newcommand{\pderiv}[2]{\frac{\partial}{\partial #1} (#2)}

% Integral dx
\newcommand{\dx}{\mathrm{d}x}

% Alias for the Solution section header
\newcommand{\solution}{\textbf{\large Solution}}

% Probability commands: Expectation, Variance, Covariance, Bias
\newcommand{\E}{\mathrm{E}}
\newcommand{\Var}{\mathrm{Var}}
\newcommand{\Cov}{\mathrm{Cov}}
\newcommand{\Bias}{\mathrm{Bias}}

\begin{document}

\maketitle

\pagebreak

\begin{homeworkProblem}
Let $f:A\to B$ and let $g:B\to C$ be functions, and let $h=g\circ f$. Determine which of the following statements are true. Give proofs of the true statements and counterexamples for the false statements.

(a) If $h$ is injective, then $f$ is injective.

(b) If $h$ is injective, then $g$ is injective.

(c) If $h$ is surjective, then $f$ is surjective.

(d) If $h$ is surjective, then $g$ is surjective.\\

\textbf{Proof:}\\

Consider $h$ is injective, for any $a_i\not=a_j\in A, h(a_i)\not=h(a_j)$\\

(a) \textbf{TRUE}. \textbf{Proof by contradiction:} Suppose $f$ is not injective, then $\exists\ a_i\not=a_j$ but $f(a_i)=f(a_j)$, then \[h(a_i)=g(f(a_i))=g(f(a_j))=h(a_j).\]

which contradicts the fact that $h$ is injective.

So statement (a) is true.
\qed\\

(b) \textbf{FALSE}. \textbf{Counterexample:} Suppose $A=\{a\}, B=\{b_1, b_2\}, C=\{c\},$ then $f(a) = b_1,$ but $g(b_1)=g(b_2)=c$. Clearly, $f, h$ are injective, but $g$ is not injective as $b_1\not=b_2$.

So statement (b) is false.

\qed\\

Consider $h$ is surjective now, $\forall c \in C, \exists\ a \in A$ such that $h(a)=c$.\\

(c) \textbf{FALSE}. \textbf{Counterexample:} Suppose $A=\{a_1,a_2\}, B=\{b_1, b_2\}, C=\{c\}$, then $f(a_1)=f(a_2)=b_1, g(b_1)=g(b_2)=c$. Then $h(a_1)=h(a_2)=c$, $h$ is surjective. But $f$ is not surjective, since there is not an $a\in A$ such that $f(a)=b_2.$

So statement (c) is false. 

\qed\\

(d) \textbf{TRUE}. Suppose $a\in A, h(a)\in C,$ then $h(a)=g(f(a)).$ Since $h$ is surjective, the image of $h$ covers everything in $C$. Besides, the image of $h$ is a subset of the image of $g$, so the image of $g$ contains everything in $C$ as well, i.e. $g$ is surjective.

\qed \\

\end{homeworkProblem}

\pagebreak

\begin{homeworkProblem}[3]
Recall that $[n]=\{1,2,\dots,n\}$. Let $A$ denote set of subsets of $[n]$ with an even number of elements, and let $B$ denote the set of subsets of $[n]$ with an odd number of elements. Prove that $|A|=|B|$ by constructing an explicit bijection from $A$ to $B$.\\

\textbf{Proof:}\\

First we have to claim that $n\not=0$ since if $[0]=\{\emptyset\}$ then $A=\{\emptyset\}, B=\emptyset$, the statement is just trivially false.\\

Then, $\forall n\in \mathbb{N}, n\geq 1$, we construct the following interesting function: $\forall a\in A,$

\[
f(a)=
\begin{cases}
	f(a) \text{\textbackslash} \{1\}\ &\text{ when } 1\in a,\\
	f(a)\cup\{1\}\ &\text{ when } 1\not\in a.
\end{cases}
\]

This means, if $a$ is an element of $A$, i.e. $a$ is a subset of $[n]$ with even number of elements, then we can find a way to map it to a unique element of $B$ (with odd number of elements).\\

The easiest way is just add/remove an element to/from $a$, so that the total number of elements becomes odd. And we can set up a criteria for this:

\begin{itemize}
	\item If $a$ has a certain element, then we remove it from $a$.
	\item If $a$ does not have such certain element, then we append it to $a$.
\end{itemize}

Actually, this certain element could be any element in the powerset of $[n]$. For simplicity, we such select $\{1\}$.\\

Now we claim $f: A\to B$ is a bijection.

\begin{itemize}
	\item Suppose $x\not= y, x,y\in A$.
	\begin{itemize}
		\item If only one of $x,y$ contains $\{1\}$. WLOG, say $\{1\}\subseteq x$, then $\{1\}\not\subseteq f(x), \{1\}\subseteq f(y).$ Therefore, $f(x)\not=f(y)$, $f$ is injective in this case.
		\item If $x,y$ both contain $\{1\}$, then neither of $f(x), f(y)$ has $\{1\}$, but still the rest part $f(x)-\{1\}\not=f(y)-\{1\}.$ $f$ is injective in this case as well.
		\item Similarly, if $x,y$ both don't have $\{1\}$, then both $f(x), f(y)$ do have $\{1\}$, but stil $f(x)\cup\{1\}\not=f(y)\cup\{1\}.$ $f$ is injective.
	\end{itemize}
	\item $\forall\ b\in B,$ we can always go backward and find an $a$ such that $f(a)=b.$ The idea is intuitive: just check if $b$ contains $\{1\}$. In this way, $f$ is surjective.
\end{itemize}

Hence, $f: A\to B$ is a bijection.

\qed

\end{homeworkProblem}

\pagebreak

\begin{homeworkProblem}
Construct explicit bijections: $f: (0, 1)\to [0, 1)$ and $g:(0,1)\to [0, 1]$.\\

\textbf{Solution:}\\

Consider the function:

\[
f(x) = 
\begin{cases}
	0, &x=\frac{1}{2};\\
	\frac{1}{x^{-1}-1}, &x=\frac{1}{n}\ \text{for}\ $n=3,4,5,\dots$;\\
	x, &\text{ otherwise.}
\end{cases}
\]

Again, consider the function:

\[
g(x) = 
\begin{cases}
	0, &x=\frac{1}{2};\\
	1, &x=\frac{1}{3};\\
	\frac{1}{x^{-1}-2}, &x=\frac{1}{n}\ \text{for}\ $n=4,5,6,\dots$;\\
	x, &\text{ otherwise.}
\end{cases}
\]

The idea of constructing these two bijections is to use some fixed point value in the domain to map to the boundary value in the image (in $f$, we use $f(\frac{1}{2})=0$; in $g$, we use $g(\frac{1}{2})=0, g(\frac{1}{3})=1.$ Then the values of $f(\frac{1}{n})$ and $g(\frac{1}{n})$ are just shifted to $\frac{1}{n-1}$ and $\frac{1}{n-2}$. And the rest part of the function remains as $f(x)=x$ (or $g(x)=x$).\\

$f$ and $g$ are obviously injective, since not a number in the image was hit twice. \\

They are also surjective, since every number in the image is covered. \\

(\textit{Recall:} These two functions are just the variations of an example in week 4 tutorial.)

\end{homeworkProblem}

\pagebreak

%
% Non sequential homework problems
%

% Jump
\begin{homeworkProblem}
Let $L$ be the set of all sentences of the English language. Prove that $L$ is countable. (For the purpose of this exercise, a sentence of the English language is any finite sequence of characters chosen from the set of characters visible on your computer's keyboard.)\\

\textbf{Proof:}\\

Consider $A = \{\text{All characters visible on the computer's keyboard}\},$ which is trivially countable.\\

Now we claim the set of all finite sequences of elements of $A$, which is the set of all sentences of the English language i.e. $L$, is also countable.\\

Suppose $\forall n\in \mathbb{N}, L_n=\{\text{All sequences of length $n$ of elements of $A$}\}$, we can prove this by induction.\\

\textbf{Base step:} When $n=0$, $L_0=\{\emptyset\}$. Obviously, $L_0$ is countable.\\ 

\textbf{Inductive step:} Suppose $k\in\mathbb{N}^+$ and $L_k$ is countable, we want to show 

\[L_{k+1}=\{\text{All sequences of length $k+1$ of elements of $A$}\}\]

 is also countable.\\

Consider the function $F:L_{k}\times A\to L_{k+1}$ as 

\[F(f,a)=f\cup\{a\},\]

where $f\in L_k$ is a sequence of $k$ elements of $A$, and $a\in A$ is an element of $A$. In fact, $F(f,a)$ is a sequence of length $k+1$ starting with sequence $f$ and end with $a$ as its $(k+1)$th term in the sequence.\\

And such $F$ is a bijection, because\\

\begin{itemize}
	\item If $(f_1, a_1) \not= (f_2, a_2)$, then $f_1\cup\{a_1\}\not=f_2\cup\{a_2\}$. $F$ is injective.
	\item For all element in $L_{k+1}$, it can be decomposed into two parts: a sentence of length $k$ as $f$, which is in $L_k$, and an element of $A$ as $a$. $F$ is surjective.\\
\end{itemize}

Therefore, the following two sets should have the same cardinality as:

\[|L_k\times A|=|L_{k+1}|\]

By theorem that \textbf{the cartesian product of 2 countable sets is also countable,} here both $L_k$ and $A$ are countable, so $L_{k+1}$ should his countable too.\\

Therefore, for any finite length $n$, $L_n$ is a countable set. Hence $L$ is countable.\\

\qed

\end{homeworkProblem}

\end{document}
