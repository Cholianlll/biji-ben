\documentclass[a4paper, 11pt, twoside]{article}
\usepackage{amssymb}
\usepackage{amsmath}
\begin{document}

2017-02-23-lec02
\\

A \textit{set} is a well-defined collection of objects.

We call the objects of a set its elements or members.

If $A$ is a set, we write $x\in A$ to mean that $x$ is a member of $A$.

If $x \not\in A$, $x$ is not a member of $A$.

If $A, B$ are sets, and every element of $A$ is an element of $B$, then we say $A$ is a subset of $B$. We write $A \subseteq B$.

\[ 	\mathbb{R} \supseteq
	\mathbb{Q} \supseteq
	\mathbb{Z} \supseteq
	\mathbb{N} \]
	
Last we saw that $\sqrt{2} \not\in \mathbb{Q}$, but $\sqrt{2}\in \mathbb{R}$. That implies $\mathbb{R} \not= \mathbb{Q}$.

We can describe subsets of given set $A$ by writing $\{x\in A: x \text{satisfies some conditions}\}$

Let $A$ be all cities in Australia, $\{x \in A: x \text{has a population of at least 3 million people} = \{Sydney, Melbourne\}$

Example: Find all real numbers satisfying $x^2 < x$.

\[
\begin{split}
	x^2 &< x \\
	x(x-1) &< 0 \\
	\text{exactly one of } x, x-1 \text{is smaller than}   0\\
	0 < x &< 1
\end{split}
\]

$\{x \in \mathbb{R}: x^2<x\}=\{x\in \mathbb{R}: 0<x<1\} = (0,1)$
\\

\textbf{Coin Problem}: We have a collection of piles of coins. Define a transformation that for each time, we take one coin from each pile to make a new pile.

Question: Describe all non-empty collections of coins unchanged under this transformation.

Let $S$ be the set of collections of coins which are unchanged by transformation.

Let $T$ be the set of all collections consisting of one pile of size $1$, one of size $2$, $\dots$, one of size $n$ for some natural number $n$. (Since we are referring to collections, order does not matter.)
\\

Claim $S=T$. (must show $S\subseteq T$ and $T \subseteq S$.)

\textbf{Proof:}

First let's show $T\subseteq S$.

given a collection in $T$ with piles of sizes of $1, 2, \dots , n$ for some fixed integer $n$. When we transform, we get piles of sizes of $0, 1, \dots , n-1$ plus one new pile of size $n$.

The result is a collection of piles of sizes $1, 2, \dots ,n$, exactly as we started with.

Thus this collection is unchanged by transformation and hence in $S$.
\\

Next, we show $S\subseteq T$.

Let $a \in S$, i.e. $a$ is a collection unchanged by transformation.

Let $m$ be the number of piles in $a$.

Observe that when we transform a collection with $m$ piles, we also create a new pile of size $m$. If $a$ is unchanged by transformation, then $a$ must have a pile of size $m$ to begin with.

Now that we know $a$ has a pile of size of $m$. We see that the transformation of $a$ must have a pile of size $m-1$. But since $a$ is unchanged under transformation, this implies $a$ itself must have a pile of size $m-1$.

Continuing with this reasoning we see that $a$ must have piles of size $m, m-1, m-2, \dots ,1$. But $a$ only has $m$ piles, so these are all the piles of $a$. Thus $a$ has the desired form and $a\in T$. 

\end{document}