\documentclass[a4paper, 11pt, twoside]{article}
\usepackage{amssymb}
\usepackage{amsmath}
\begin{document}
2017-02-24-lec03
\\

\textbf{Operations with sets} \\

\textbf{Product:} Given set $S, T$, their product is defined as $S\times T:\{(x,y): x \in S, y \in T\}$. e.g. $S^k = S\times S \times \cdots \times S = \{(x_1, \dots, x_k): x_i \in S\}$

Example: Cartesian Plane $\mathbb{R}^2 = \{(x,y): x, y \in \mathbb{R}$, also notated as $(\mathbb{R} \times \mathbb{R})$.

$\mathbb{R}\times \mathbb{Q} \subseteq \mathbb{R}^2$

$\mathbb{Q}\times \mathbb{R} \subseteq \mathbb{R}^2$

$\mathbb{Q}^2 \subseteq \mathbb{R}\times \mathbb{Q}$

$\mathbb{Q}^2 \subseteq \mathbb{Q}\times \mathbb{R}$\\

\textbf{Power set:} given a set $S$, the power set of $S$ is defined as $2^{S} = \{T | T \subseteq S\}$

\textbf{Remark:} There is a unique set with no elements called the empty set $\emptyset$. Note that $\emptyset$ is a subset of every subset.

$S=\{Sydney, Melbourne\}$

$2^S = \{\emptyset, \{Sydney\}, \{Melbourne\}, S\}$

More generally, if $S$ is a finite set with $n$ elements, then $2^S$ has $2^n$ elements .

$S = \{x_1, \dots , x_n\}, T\subseteq S?$

Either $x_1 \in T$ or $x_1 \not \in T$,

$x_2 \in T$ or $x_2 \not \in T, \dots$

$x_n \in T$ or $x_n \not \in T.$ All together, $2^n$.\\

\textbf{Unions, intersections, difference, complement}

Suppose $A, B \subseteq U, A\cup B= \{x \in U | x \in A \text{ or } x \in B\}.$

$A\cap B = \{x\in U | x\in A \text{ and } x \in B\}.$ 

$A-B = \{x\in U|x\in A \text{ and } x \not \in B\}.$

$A^C = \{x \in U |x \not \in A\}.$

Example: 

$E=\{2k: k\in \mathbb{Z}\}$ even integer.

$O=\{2k+1: k\in \mathbb{Z}\}$ odd integer.

$E \cup O = \mathbb{Z}$

$E \cap O = \emptyset$

$E^C = O$

$O^C = E$

$E - O = E$\\

\textbf{Functions:} Let $A, B$ be sets, a function $f: A \rightarrow B$ assigns to each $a \in A$ an element $f(a) \in B$. We call $A$ the domain of $f$, $B$ the target of $f$.

$\forall S \subseteq A$, can consider $f(S):= \{f(a): a \in S\} \subseteq B$

We call $f(A)$ the image of $f$, or the range of $f$.

Example: Let $[n] = \{1, 2, \dots , n\}$. How many distinct functions from $[n]$ to $[n]$?

$n$ choices for $f(1)$, $n$ choices for $f(2)$, $\cdots$, $n$ choices for $f(n)$.

$n^n$ possible functions\\

The \textbf{graph of a function} $f:A\rightarrow B$ is the subset $\{(a, f(a)):a\in A\}\subseteq A\times B$. In fact, $f$ is completely determined by its graph, i.e., I could equivalently define a function $f: A\rightarrow B$ to be a subset $S\subseteq A \times B$ such that $\forall a \in A, \exists$ a unique $b\in B$ such that $(a, b) \in S$.\\

\textbf{Reading:} Chapter 1 (skip quadratic formula and arithmetic/geometric inequality) sets, functions...
\end{document}