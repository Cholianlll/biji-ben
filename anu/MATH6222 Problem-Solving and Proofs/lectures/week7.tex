\documentclass[a4paper, 11pt, twoside]{article}
\usepackage{amssymb}
\usepackage{amsmath}
\begin{document}
\title{MATH6222 Week 7}
\author{xarskii}
\date{2017-04-20}

\maketitle

\section{Modular Arithmetic}

This is Chapter 7 in textbook.

\paragraph{Problem:} Find the last digit of $971^{216}+523^{121}$\\

\textbf{Observation:}

\begin{itemize}
	\item The last digit of $a\times b$ only depends on last digit of $a$ and the last digit of $b$.\\
	(So the last digit of $971^{216}$ is still $1$)\\
	($3^{121}=3^{120}\cdot 3=(3^4)^{30}\cdot 3=(81)^{30}\cdot 3$. The last digit must be $1\times 3= 3$.)
	\item The last digit of $a+b$ only depends on the last digit of $a$ and $b$.
\end{itemize}

Therefore the last digit of the sum is $4$.\\

\textbf{Definition:} Fix a natural number $n$, called the modulus. We say $a,b\in\mathbb{Z}$ are \textbf{congruent modular $n$} if $n|(a-b)$. Equivalently, we can write, $a=b+kn$ for some $k\in\mathbb{Z}$. We write this as $a\equiv b\mod n.$\\

\textbf{Example:}\\

$\mod 3: \dots\dots, -3, -2, -1, 0, 1, 2, 3, 4, \dots\dots$\\

$\{a\in\mathbb{Z}: a\equiv 0 \mod 3\}$ is the set of all multiples of $3$, i.e. $3k$

$\{a\in\mathbb{Z}: a\equiv 1 \mod 3\}$ is the set of all integers of the form $3k+1$

$\{a\in\mathbb{Z}: a\equiv 2 \mod 3\}$ is the set of all integers of the form $3k+2$\\

$a\equiv b \mod 10 \iff a$ and $b$ have same last digit. 

Having the same last digit is the same as saying that $a$ and $b$ differ by a multiple of $10$?

But, $\{a\in\mathbb{Z}: a\equiv 3\mod 10\}=\{\dots, -17, -7, 3, 13, 23, \dots\}$ So the previous statement is almost true.\\

Given $a\in\mathbb{Z}$, the congruence class of $a$ modulo $n$ is 

\[\bar{a}=\{x\in\mathbb{Z}: a\equiv x \mod n\}\]

For $\mod 3$, 

$\bar{0}=\{\dots, -6, -3, 0, 3, 6, \dots\}$

$\bar{1}=\{\dots, -5, -2, 1, 4, 7, \dots\}$

$\bar{2}=\{\dots, -4, -1, 2, 5, 8, \dots\}$

Actually, $Z=\bar{0}\cup\bar{1}\cup\bar{2}.$\\

\paragraph{Proposition:} Working $\mod n$, there are exactly $n$ distinct congruence classes. And every integer lies in exactly one of these classes.\\

For $\mod 2, \bar{0}$ is the set of even numbers, $\bar{1}$ is the set of odd numbers. Two congruence classes.

Also, $\bar{0}=\bar{2}$. Same set.

\paragraph{Proof:} Observe that $\bar{0}, \bar{1}, \dots, \overline{n-1}$ are clearly distinct congruence classes.

Reason: If $0\leq i < j\leq n-1$, then $i-j$ can't be divisible by $n$. (too close to each other)

Every other integer lies in exactly one of these congruence classes ($\mod n$).

Why? Given any integer $a$, division algorithm tells you that $\exists\ !k, r\in\mathbb{Z}$ such that $a=kn+r$ where $0\leq r\leq n-1.$ And this is the same as $a\equiv r\mod n$. (Note: ! means ``unique'')

\paragraph{Key Lemma of Modular Arithmetic:} If $a\equiv a' \mod n$ and $b\equiv b' \mod n$, then

\begin{enumerate}
	\item $a+b\equiv a'+b' \mod n$
	\item $ab\equiv a'b' \mod n$
\end{enumerate}

$\mod 10$: last digit of sum or product only depends on last digits of summands.

$\mod 2$: (oddness/eveness) the parity of sum/product only depends on the parity of inputs.

\paragraph{Proof:} $a=a'+kn$ for some $k\in\mathbb{Z}$. $b=b'+ln$ for some $l\in\mathbb{Z}$.

Then $a+b=a'+kn+b'+ln=a'+b'+(k+l)n$. This by definition means $a+b\equiv a'+b' \mod n.$

If we multiple $a, b$,

$ab=(a'+kn)(b'+ln)=a'b'+n(kb'+la'+kln)\equiv a'b' \mod n.$

We are done.

\section{Friday's Lecture}\

Last time:

$a\equiv b\mod n\iff a=b+kn$ for some $k\in\mathbb{Z}$.

$\overline{a}=\{x\in\mathbb{Z}: x=a\mod n\}$

$\overline{0}=\{kn: k\in\mathbb{Z}\}$

$\overline{1}=\{kn+1: k\in\mathbb{Z}\}$

$\overline{2}=\{kn+2:k\in\mathbb{Z}\}$

$\overline{n-1} = \{kn+(n-1): k\in\mathbb{Z}\}$

Every integer is in exactly one of these congruence classes.

For any integer $m$, $\exists$ unique $k,r $ such that $m=kn+r, 0\leq r\leq n-1$.

\begin{enumerate}
	\item Shallow: we can be clever about ``reducing an integer mod $n$".
	\item Deep: addition/multiplication is well-defined on congruence  classes, which \textbf{means finding the remainder often division by $n$, or which of these congruence classes it's in.}
\end{enumerate}

Example: Find last digit of $971^{216} + 513^{121}$. I'm asking to reduce this number $\mod 10$.

$971\equiv 1\mod 10$

$971^{216}\equiv 1^{216}\mod 10\equiv 1\mod 10$

\[
\begin{split}
	523^{121}&\equiv 3^{121}\mod 10\\
	&\equiv (3^{4})^{30}\cdot 3 \mod 10\\
	&\equiv 81^{30}\cdot 3\mod 10\\
	&\equiv 1^{30}\cdot 3\mod 10\\
	&\equiv 3\mod 10
\end{split}
\]

$971^{216}+523^{121}\equiv 1+3\mod 10\equiv 4\mod 10$\\

Example: Suppose it's 3 o'clock. What time will show on a 12-hour clock after $47^{101}$ hours.

Reduce $3+47^{101}\mod 12$.

$47^{101}\equiv 11^{101} \mod 12$

$11\equiv -1 \mod 12$ or $11^2\equiv 121\equiv 1 \mod 12$

$11^{101} \equiv (-1)^{101} \mod 12 \equiv -1 \mod 12$\\

Example: $9,18,\dots$ the sum of the digits of the multiple of $9$ is itself a multiple of $9$.

\[
\begin{split}
	a_n, a_{n-1}, \dots, a_1, a_0 &= a_n\times 10^n + a_{n-1}\times 10^{n-1} + \cdots + a_1\times 10 + a_0 \\
	&\equiv a_n\times 1^n + a_{n-1}\times n^{n-1} +\cdots + a_1\times 1 + a_0 \mod 9
\end{split}
\]\\

$z_n:=$ set of congruence classes $\mod n$ = $\{\bar{0}, \bar{1}, \bar{2}, \dots, \overline{n-1}\}$\\

$\overline{a}+\overline{b}=\overline{a+b}$

$\overline{a}\cdot\overline{b}=\overline{a\cdot b}$\\

$z_3 = \{\overline{0}, \overline{1}, \overline{2}\}$

$\overline{0}=\{3,6,9,\dots\}$

$\overline{1}=\{4,7,10,\dots\}$

$\overline{2}=\{5, 8, 11,\dots\}$

$\overline{1}+\overline{2}=\overline{0}$\\

$z_4=\{\overline{0}, \overline{1}, \overline{2}, \overline{3}\}$

\begin{table}[htbp!] 
	\centering
	\begin{tabular}{|c|c|c|c|c|}
		\hline
		$+$ & $\overline{0}$ & $\overline{1}$ & $\overline{2}$ & $\overline{3}$ \\
		\hline
		$\overline{0}$ & $\overline{0}$ & $\overline{1}$ & $\overline{2}$ & $\overline{3}$ \\
		\hline
		$\overline{1}$ & $\overline{1}$ & $\overline{2}$ & $\overline{3}$ & $\overline{0}$ \\
		\hline
		$\overline{2}$ & $\overline{2}$ & $\overline{3}$ & $\overline{0}$ & $\overline{1}$ \\
		\hline
		$\overline{3}$ & $\overline{3}$ & $\overline{0}$ & $\overline{1}$ & $\overline{2}$ \\
		\hline
	\end{tabular}
\end{table}

\begin{table}[htbp!] 
	\centering
	\begin{tabular}{|c|c|c|c|c|}
		\hline
		$\times$ & $\overline{0}$ & $\overline{1}$ & $\overline{2}$ & $\overline{3}$ \\
		\hline
		$\overline{0}$ & $\overline{0}$ & $\overline{0}$ & $\overline{0}$ & $\overline{0}$ \\
		\hline
		$\overline{1}$ & $\overline{0}$ & $\overline{1}$ & $\overline{2}$ & $\overline{3}$ \\
		\hline
		$\overline{2}$ & $\overline{0}$ & $\overline{2}$ & $\overline{0}$ & $\overline{2}$ \\
		\hline
		$\overline{3}$ & $\overline{0}$ & $\overline{3}$ & $\overline{2}$ & $\overline{1}$ \\
		\hline
	\end{tabular}
\end{table}


\end{document}