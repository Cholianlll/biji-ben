\documentclass[a4paper, 11pt, twoside]{article}
\usepackage{amssymb}
\usepackage{amsmath}
\begin{document}
\title{MATH6222 week 4 lecture 10}
\author{Rui Qiu}
\date{2017-03-16}

\maketitle

If $A$ is a finite set, let $|A|$ denotes the number of elements in $A$.

Claim that

\[
\begin{split}
	|A|&=|B| \iff \exists \text{ a bijection } f: A\to B.\\	
	|A|&\leq |B| \iff \exists \text{ a injection } f: A\to B.\\
	|A|&\geq |B| \iff \exists \text{ a surjection } f: A \to B.\\
\end{split}
\]

\paragraph{Proof of the first:}\ \\

Suppose $|A|=|B|$, then $A=\{a_1, a_2, \dots, a_n\}, B=\{b_1,b_2,\dots, b_n\}.$

Then define $f:A\to B$, i.e. $f(a_i)=b_i.$ Clearly $f$ is a bijection (which maps every $a_i$ to $b_i$).

On the other hand, if $f:A\to B$ is a bijection.

Then I claim $A$ and $B$ have the same number of elements.

Let's write $A=\{a_1, \dots, a_n\}.$

Then $B=\{b_1,\dots, b_n\}.$

--------------------

What does it mean to say \textit{$A$ has $n$ elements}?

It means we can write $A$ as $\{a_1, a_2, \dots, a_n\}$

--------------------

Given 2 arbitrary sets $A, B$, we'll say that $A$ and $B$ have the same cardinality (yeah i'm fancy) and write $|A|=|B|$ if $\exists$ bijection $f:A\to B$.\\

\paragraph{Proposition:}
\begin{enumerate}
	\item For any set $A, |A|=|A|.$ (id. $A\to A$)
	\item For any set $A, B, |A|=|B| \iff |B| = |A|.$ ($f:A\to B, f^{-1}:B\to A$)
	\item For any set $A, B, C$, if $|A|=|B|, |B|=|C| \implies |A|=|C|.$ ($f:A\to B, g: B\to C, g\circ f:A\to C$)
\end{enumerate}

\paragraph{Remark:}\ \\

Given $f: A\to B, g: B \to C,$ we can define

\[g\circ f: A\to C\]

\paragraph{Proposition:} If $f: A\to B$ and $g: B\to C$ are injective/surjective, then $g\circ f$ is injective/surjective.\\

\paragraph{Proof:}\ \\
$g\circ f: A\to C$ Given $a_1, a_2\in A$ with $a_1\not= a_2$, must show $(g\circ f)(a_1)\not=(g\circ f)(a_2).$

Since $f$ injective, $f(a_1)\not=f(a_2)$ (as elements of $B$).

Since $g$ injective, $g(f(a_1))\not=g(f(a_2))$.

For surjections...\\

\textbf{Example:}

$\mathbb{N}=\{1, 2, 3, \dots\}$

$\mathbb{N}$ \textbackslash $\{1\} = \{2, 3, 4, \dots\}$

$|\mathbb{N}| = |\mathbb{N}$ \textbackslash $\{1\}|$\\

\textbf{Next example:}

$\mathbb{N}=\{1,2,3,\dots\}$

$\mathbb{E}=\{2, 4, 6,\dots\}$

$\exists$ a bijection $\mathbb{N}\to\mathbb{E}$\\

$\mathbb{N}\subset \mathbb{Z} \subset \mathbb{Q} \subset \mathbb{R}$

\textbf{Example:}

$\exists$ bijection $\mathbb{N}\to\mathbb{Z}, n \to \left.
\begin{cases}
	\frac{n}{2}, &\text{ even}\\
	-\frac{n-1}{2}, &\text{ odd}
\end{cases}
\right.$\\

\paragraph{Definition:} We say a set $A$ is \textbf{countably infinite} if $|A|=|\mathbb{N}|$.\\

$\mathbb{Q}^+ =\{\frac{a}{b}: a, b \in \mathbb{N} \text{ st. } a \text{ and } b \text { have no common factors}$\\

\[
\begin{split}
	&\frac11, \frac21, \frac31, \frac41, \frac51, \dots\\
	&\frac12, \frac32, \frac52, \frac72, \frac92, \dots\\
	&\frac13, \frac23, \frac43, \frac53, \frac73, \dots\\
	&\frac14, \frac34, \dots\\
\end{split}
\]

Then ... zig-zag...\\

What we really need to show here is to show $\mathbb{N}\times\mathbb{N}\to \mathbb{N}$

Think about this: $(i, j)\to 2^{i-1}(2j-1)$\\

$\mathbb{Q}=\mathbb{Q}^+\cup\mathbb{Q}^-\cup\{0\}$

do $\{0, \frac{1}{1}, -\frac{1}{1}, \frac{2}{1}, -\frac{2}{1}, \frac{1}{2}, -\frac{1}{2}, \dots\}$

\end{document}