\documentclass[a4paper, 11pt, twoside]{article}
\usepackage{amssymb}
\usepackage{amsmath}
\begin{document}
\title{MATH6222 Week 11 Lecture Notes}
\author{Rui Qiu}
\date{2017-05-15}

\maketitle

\section{Monday's Lecture}

\paragraph{Eulerian Circuit:} Every edge exactly once, starts and ends at the same vertex.

\paragraph{Eulerian Path:} Every edge exactly once, start and ending can be different.

\begin{enumerate}
	\item If there is a vertex with an odd number edges, then there cannot be Eulerian circuit.
	\item Out graph should be connected.
\end{enumerate}

\paragraph{Remaining Question:} If all vertices have even edges, then must I have an Eulerian circuit?

\textbf{A graph $G$ consists of a vertex set $V(G)$ (finite set), and edge set $E(G)$ (finite set).}\\

\[I_G:\ E(G)\rightarrow V(G):=\{\text{set of unordered pairs of elements of } V(G)\}
\]

edge $\rightarrow$ pair of vertices it connects.\\

A graph is simple if it is injective.\\

$V(G)=\{a,b,c,d\}$

$E(G)=\{e_1,e_2,\dots, e_6\}$

$i_G(e_1)= \{a, b\}=i_G(e_2)$

$i_G(e_3)=i_G(e_4)=\{a, d\}$

$i_G(e_5)=\{b, c\} \dots \dots $\\

\paragraph{Alternative definition:} A graph $G$ is a set $V(G)$ together with a symmetric relation $E(G)\subseteq V(G)\times V(G) \implies$ doesn't tell the number of edges between $2$ edges.\\

A graph, if $i_G$ is injective, is simple.

Geometrically: simple graph has no multiple edges, no loops.\\

If $i_G(e)=\{u, v\}$, then $u,v$ are adjacent to edge $e$.\\

The degree of a vertex is just the number of edges adjacent to it.\\

\paragraph{Definition:} a trail in a graph $G$ is a list $v_0e_1v_1e_2v_2\dots v_k$, where $v_i\in V(G), e_i \in E(G)$, satisfies

\begin{itemize}
	\item $e_i$ is adjacent to $v_{i-1}, v_i$
	\item every $e_i$ is distinct
\end{itemize}

A trail is close if $v_o=v_k\implies$ Eulerian circuit is a closed trail.

\section{Thursday's Lecture}

Last time:\\

A \textbf{trail} is a list $v_0e_1v_1e_2v_2\dots e_kv_k$ such that 

\begin{enumerate}
	\item $v_{i-1}$ and $v_i$ are the endpoints of $e_i$
	\item $e_1,\dots, e_k$ distinct
\end{enumerate}

A trail is \textbf{closed} if $v_o=v_k$.\\

A trail is \textbf{Eulerian} if it contains all edges of $G$.\\

\paragraph{Proposition:} a graph has a closed Eulerian trail iff every vertex has even degree.

\paragraph{Proof:}\ \\

$\Longrightarrow$ Suppose $v_0e_1v_1\dots v_k$ is  a closed Eulerian trail. Let $v\in V(G)$ be any vertex (say $v\not= v_o, v_k$). Suppose $v$ shows up in this list as $v_{i_1}, v_{i_2}, \dots, v_{i_l}, 1\leq i_1\leq \dots \leq i_l\leq k$\\

The edges adjacent to $v$ in $G$ are just: $e_{i_1}, e_{i_1+1}, e_{i_2}, e_{i_2+1}, \dots, e_{i_l}, e_{i_l+1}$.\\

There is an even number of these.\\

If $v\in V(G)$ is the endpoint, i.e. $v=v_0=v_k$. Then the edges adjacent to $v$ are $e_1, e_k$ plus pairs of edges for every additional time $v$...\\

$\Longleftarrow$ Suppose every vertex has even degree. Consider a trail in $G$ of maximal length.\\

\textbf{Observation:} It must be closed!\\

Suppose $v_0e_0\dots e_kv_k$ is a maximal trail. Suppose not closed so $v_k\not=v_0$.\\

The number of edges contained in this trail and adjacent to $v_k$ must be odd.\\

Since degree of $v_k$ is even, there must be some edge adjacent to $v_k$ which has not appeared in out trail -- call it $e_{k+1}$. Then we get a longer trail by adding $e_{k+1}v_{k+1}$.\\

Now we'll show that a max length trail must be Eulerian.\\

Suppose there is some edge not contained in your trail. We can assume that this edge is adjacent to some vertex already on our trail.\\

Construct a graph $G$ by deleting all edges of $G$ that are contained in our max trail.\\

This new graph $G'$ still has every vertex of even degree.\\

Consider any maximal trail in $G'$ starting from $v_i$.\\

Get a closed trail in $G'$ starting and ending at $v_i$.\\

But now we can get a longer trail!

\[v_0e_1v_1\dots e_iv_ie_{i+1}\dots v_k\]

\[v_0e_1v_1\dots e_iv_i \longleftrightarrow v_ie_{i+1}\dots v_k\]

We conclude that a maximal trail must include every edge of $G$.\\

\paragraph{\"Uber-Problem:} A $k$-colouring of a graph $G$ is a function $f: V(G)\longmapsto [k]$ such that $f(u)\not= f(v)$ wherever $u,v$ are adjacent.\\

Let 

\begin{itemize}
	\item $P_n$ be a path with $n$ vertices.
	\item $C_n$ be a cycle on $n$ vertices.
	\item $K_n$ be a complete graph on $n$ vertices.
\end{itemize}

The chromatic number of a graph denoted $\chi(G)$, is the minimum number of colors needed to $k$-colour.

\begin{itemize}
	\item $\chi(P_n)=2$
	\item $\chi(C_n)=\begin{cases}
		3,\ &n\text{ odd}\\
		2,\ &n\text{ even}
	\end{cases}$
	\item $\chi(K_n)=n$
\end{itemize}

\paragraph{Definition:} a graph is \textbf{bipartite} if $\chi(G)=2$. Equivalently, $V(G)=X\cup Y$ such that every edge in $G$ has one endpoint in $X$, and one endpoint in $Y$.

\paragraph{Theorem:} a graph $G$ is bipartite iff it has no cycles of odd length.\\

A cycle is a list except we insist $v_0=v_k$.\\

\paragraph{Definition:} a connected graph with no cycles is called a \textbf{tree}.
\end{document}