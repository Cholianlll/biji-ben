\documentclass[a4paper, 11pt, twoside]{article}
\usepackage{amssymb}
\usepackage{amsmath}
\begin{document}
\title{MATH6222 week 6 lecture 17}
\author{Rui Qiu}
\date{2017-03-31}

\maketitle

\paragraph{Definition:} An integer $n > 2$ is \textbf{prime} if $d|n\implies d=1$ or $d=n$. We say an integer is \textbf{composite} if not prime. Equivalently, $n$ is composite if $\exists$ divisor $d|n$ with $1< d < n$.\\

\paragraph{Lemma:} Any integer $n>1$ is a product of primes, i.e.

\[n=p_1p_2\cdots p_k\]

for some prime $p_1,p_2,\dots, p_k.$\\

\paragraph{Proof:} Prove this by strong induction on $n$.

Base Case $n=2$. ($2$ is a product of $2$, which is prime itself.)

Inductive Step: Given integer $n$, either $n$ is prime or $n$ is composite.

If $n$ is prime, nothing to prove.

If $n$ is composite, can write $n=n_1n_2$ where $1 < n_1 < n, 1< n_2 < n.$

By the induction hypothesis, 

\begin{itemize}
	\item $n_1=p_1p_2\cdots p_k,\ p_i$ prime.
	\item $n_2=q_1q_2\cdots q_l,\ q_j$ prime.
\end{itemize}

So $n=n_1n_2=p_1\cdots p_kq_1\cdots q_l.$\\

\paragraph{Fundamental Theorem of Arithmetic:} Any integer $n>2$ can be written uniquely as a product of primes.

Suppose $n=p_1p_2\cdots p_k$, and $n=q_1q_2\cdots q_l$. Then must have $k=l$, after reordering we have $p_1=q_1, p_2=q_2, \dots$\\

Imagine a world with only even numbers. Define the ``new prime`` as numbers non-divisible by smaller even numbers, in this case $6$ and $10$ are primes, etc.

\begin{itemize}
	\item Determine which numbers $\leq 40$ are prime.\\
	$2, 6, 10, 14, 18, 22, 26, 30, 34, 38$.
	\item Determine prime factorizations for all integers $\leq 40$.\\
	$2=2, 4=2\times 2, 6=6, 8=2\times 2\times 2, 10=10, 12 = 2\times 6, 14=14, 16=2\times 2\times 2 \times 2, 18=18, 20=2\times 10, 22=22, 24= 2\times 2\times 6, 26=26, 28=2\times 14, 30 = 30, 32=2\times 2\times 2\times 2\times 2, 34=34, 36=2\times 18=6\times 6, 40 = 2\times 2\times 10.$
\end{itemize}

Then a problem emerges, \textbf{$36$ has two ``prime`` factorizations!}

\[36=6\times 6=2\times 18\]\\

Two integers $a$ and $b$ are relatively prime if $\gcd(a,b)=1.$ (Example: $a=6, b=25, \gcd(a,b)=1.$) Equivalently, $\exists\ m,n \in\mathbb{Z}$ such that $ma+nb=1$.

(What is the relationship between this and prime factorization?)

\paragraph{Lemma:} Let $p$ be prime, let $a$ be any integer, either $p|a$ or $p$ and $a$ are relatively prime.

\paragraph{Proof:} Consider $\gcd(a,p)$. Must have

\begin{itemize}
	\item $\gcd(a,p)=1\implies a$ and $p$ are relatively prime.
	\item $\gcd(a,p)=p\implies p|a$.
\end{itemize}

\paragraph{Proposition (Key Property of Primes):} $p$ prime, $a,b$ integers. If $p|ab$ then $p|a$ or $p|b$.

\paragraph{Proof:} If $p|a$, nothing to prove. So assume $p\not|\ a.$ So $p$ and $a$ are relatively prime.

By the Euclidean algorithm, $\exists\ m,n\in\mathbb{Z}$ such that $ma+np=1$.

Let's multiple this by $b$:

\[
\begin{split}
	mab+npb=b
\end{split}
\]

$p$ divides $mab$, and $p$ divides $npb$ automatically, so $p$ divides $b$.\\

\paragraph{Corollary:} If $p|(a_1\cdots a_k)$, then $p|a_i$ for some $i$.

Proof by induction on $k$. $k=2$ done. ........... (skipped)

\paragraph{Proof of Fundamental Theorem of Arithmetic:} By strong induction on $n=2$. Suppose we have two prime factorizations of $n$:

\begin{itemize}
	\item $n=p_1p_2\cdots p_k,\ p_i$ prime.
	\item $n=q_1q_2\cdots q_l,\ q_j$ prime.
\end{itemize}

So \[p_1p_2\cdots p_k=q_1q_2\cdots q_l\]

$p_1|(q_1\cdots q_l)\implies p_1|q_i$ for some $i=1,\dots, l$.

Since $q_i$ is prime, $p_1=q_i$. After reordering, assume $p_1=q_1$.

Now we have 

\[p_2\cdots p_k=q_2\cdots q_l\]

Now by the induction hypothesis, after reordering, we have $p_2=q_2, p_3=q_3,\dots p_k=q_l$\\

\paragraph{Corollary:}

\begin{enumerate}
	\item An integer $d|n \iff $ every prime factor $d$ is a prime factor of $n$.
	\item $\gcd(a,b)$ is just product of all primes occurring in both $a$ and in $b$.
\end{enumerate}

\paragraph{Proof:}

Suppose $d|n$, then $n=dk$, then $d=p_1p_2\cdots p_i, k=q_1q_2\cdots q_j.$ .............

\end{document}