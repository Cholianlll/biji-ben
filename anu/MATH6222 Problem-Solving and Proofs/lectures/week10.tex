\documentclass[a4paper, 11pt, twoside]{article}
\usepackage{amssymb}
\usepackage{amsmath}
\begin{document}
\title{MATH6222 Week 10 Lecture Notes}
\author{Rui Qiu}
\date{2017-05-08}

\maketitle

\section{Monday's Lecture}

$[15]$

\paragraph{Qustion:} What's the largest possible subset of $[15]$ that does not have a triple of consecutive elements.

$\{1, 2, 4, 5, 7, 8, 10, 11, 13, 14\}$

\paragraph{Claim:} $10$ is largest. Need to show $11$ is impossible.\\

If we take a subset of size $11$ one of these 5 groups must have at least $3$ elements.

\paragraph{Pigeonhole Principle:} If more than $k\cdot n$ objects are divided into $n$ classes, then one class must have more than $k$ objects.

\paragraph{Proof:} Suppose every class had $\leq k$ objects. Then the total number of objects would have to be $\leq n\cdot k$.\\

Let's construct a large collection of integer points whose midpoints are not integer points (in $\mathbb{R}^2$).\\

Given $(a,b)$ and $(c,d)$. When is their midpoint an integer point?\\

\[\left(\frac{a+c}{2}, \frac{b+d}{2}\right)\]

get an integer midpoint if $(a,c)$ have same parity, $(b,d)$ have same parity.\\

\paragraph{Claim:} Given $5$ integer points on plane, some pair must have an integer midpoint.\\

Divide integer points into $4$ classes. (even, even), (even, odd), (odd, even), (odd, odd).\\

By PHP, one class must get two points. These two must have integer midpoint.\\

\[8,4,7,5,1,9,3,6,2,10\]

Can I rearrange to get rid of monotone subsequence of length $5$? Yes.

What about $4$?

\[3,2,1,6,5,4,9,8,7,10\]

\paragraph{Theorem (Erdös, 1935):} Given a sequence of $n^2+1$ integers, there exists monotone subsequence of length $n+1$.\\

Let $a_k: k =1, 2, \dots, n^2+1$ be our sequence of integers.

Let $x_k:=$ length of the largest increasing subsequence ends at $a_k$.

Let $y_k:=$ length of the largest decreasing subsequence ends at $a_k$.

\begin{table}[htbp!] 
	\centering
	\begin{tabular}{|c|c|c|c|c|c|c|c|c|c|c|}
		\hline
		$a_k$ & 7 & 4 & 8 & 5 & 1 & 9 & 3 & 6 & 2 & 10 \\
		\hline
		$x_k$ & 1 & 1 & 2 & 2 & 1 & 3 & 2 & 3 & 2 & \\
		\hline
		$y_k$ & 1 & 2 & 1 & 2 & 3 & 1 & 3 & 2 & 4 & \\
		\hline
	\end{tabular}
\end{table}

What I want to show is that one of these numbers $x_k, y_k$ must reach the value $n+1$.

Suppose this doesn't happen. Then $x_k\leq n, y_k\leq n$, so I have at most $n^2$ possible pairs $(x_k, y_k)$. Since we have $n^2+2$ pairs, some pair must be repeated: $(x_i, y_i)=(x_j, y_j), i < j$. This is impossible.

$a_i, a_j$

$(x_i, y_i)=(x_j, y_j)$.

|||||||||||||||||||||||||

\section{Thursday's Lecture}

Let$\phi(m):=$ number of $[m]$ relatively prime to $m$.

$\phi(6)=2$, $\{1,5\}$

$\phi(7)=6$, $\{1,2,3,4,5,6\}$

\paragraph{Question:} How can we compute $\phi(m)$ efficiently?

\[\phi(20), 20 = 2^2\cdot 5\]

\[\phi(20)=20-\frac{20}{2}-\frac{20}{5}+\frac{20}{10}\]

$10$ was excluded twice as both multiple of $2$ and multiple of $5$, so we have to add it back.

\[m=p_1^{e_1}p_2^{e_2},\ \phi(m)=m-\frac{m}{p_1}-\frac{m}{p_2}+\frac{m}{p_1p_2}\]

\[30=2\cdot 3\cdot 5\]

\[\phi(30)=30-\frac{30}{2}-\frac{30}{3}-\frac{30}{5}+\frac{30}{2\cdot 3} + \frac{30}{2\cdot 5} + \frac{30}{3\cdot 5} - \frac{30}{2\cdot 3\cdot 5}=30-15-10-6+5+3+2-1=8\]\\

Suppose $A_1,\dots A_n$ are subsets of a finite set $U$ (Universe). We want to count the number of elements in $U-(A_1\cup A_2\cup \cdots A_n)$

$U=[m], m =p_1^{e_1}\cdots p_k^{e_k}$.

$A_i=$ integers $\leq m$ that are multiples of $p_i$, $\phi(m)=|U-(A_1\cup \cdots A_k)|$.\\

Example: $|U|-|A_1|-|A_2|-|A_3|+|A_1\cap A_2|+|A_1\cap A_3|+|A_2\cap A_3|-|A_1\cap A_2\cap A_3|$.

\paragraph{Inclusion-Exclusion Formula:} For any subset $S=[n], A_S=\bigcap_{i\in S}A_i$,

\[|U-(A_1\cup\cdots \cup A_n)|=\sum_{S\subseteq [n]}(-1)^{|S|}|A_S|\]

where $A_{S}:=\cap_{i\in S}A_i$.

\paragraph{Proof:} First, consider $x\in U-(A_1\cup \cdots \cup A_n)$, then $x$ contributes one to the total sum because $x\in A_{\varnothing}$.

Second, consider $x\in (A_1\cup\cdots \cup A_n)$.\\

Define $T\subseteq [n]$ by:

\[i\in T \iff x \in A_i\]

Then $x\in A_S\iff S\subseteq T$.

$x$ contributes $+1$ to total sum for each $S\subseteq T$ such that $|S|=$ even 

contributes $-1$ to total sum for each $S\subseteq T$ such that $|S|=$ odd\\

The total contribution of $x$ to sum is zero.\\

If $|$number subsets of $T$ with even size$|$ = $|$number subsets of $T$ with odd size$|$. Construct an explicit bijection.\\

Compute $\phi(m):\ m=p_1^{e_1}\cdots p_k^{e_k},\ U = [m],\ A_i=$ integers $\leq m$ divisible by $p_i\ (i=1,\dots ,k)$.

\[\phi(m)=|U-(A_1\cup\cdots \cup A_k)|\]

\[A_S=\bigcap_{i\in S}A_i=\text{multiples of } \prod_{i\in S}p_i\]

\[A_{\{1,2\}}=A_1\cap A_2 = \text{multiples of } p_1p_2\]

\[\phi(m)=\sum_{S\subseteq[k]}(-1)^{|S|}|A_S|=\sum_{S\subseteq[k]}(-1)^{|S|}\frac{m}{(\prod_{i\in S}p_i)}=m\prod^k_{i=0}\left(1-\frac{1}{p_i}\right)\]

\paragraph{Definition:} A derangement is a permutation with no fixed points.

For 1,2,3: only 2,3,1 and 3,1,2 are derangements.

\paragraph{Question:} How many derangements of $[n]$?

$U=$permutations of $[n]$, $|U|=n!$.

$A_1=$permutations fixing $1$.

$A_2=$permutations fixing $2$.

$\cdots$

$A_n=$permutations fixing $n$.

Number of derangements of $[n]=\sum_{S\subseteq [n]}(-1)^{|S|}(n-|S|)!=\sum^n_{k=0}(-1)^k{n\choose k}(n-k)!=\sum^n_{k=0}(-1)^k\frac{n!}{k!}=n!\sum^n_{k=0}\frac{(-1)^k}{k!}.s$

\section{Friday's Lecture}
Last time: $A_1,\dots, A_n\subseteq U$

$|U-(A_1\cup\cdots\cup A_n)|=\sum_{S\subseteq [n]}(-1)^{|S|}|A_S|$

\paragraph{Problem:} Roll a die $n$ times. What is the probability that every number 1 to 6 appears at least one?

Counting problem... Then there are $6^n$ total possibilities. The probability of $1$ not appearing is $\left(\frac{5}{6}\right)^n$.

Let $U=\{$ be the all possible rolls of $n$ dice, length in sequences $\{1,2,3,4,5,6\}\}$.

$|U|=6^n$. Let $A_i$ be the set of rolls where $i$ never appears.

\[|U-(A_1\cup A_2\cup A_3\cup A_4\cup A_5\cup A_6)|\]

$|A_i|=S^n, A_{\{1,2\}}, A_S=\bigcap_{i\in S}A_i, |A_S|=(6-|S|)^n$

\[|U-\cdots|=\sum_{S\subseteq [6]}(-1)^{|S|}(6-|S|)^n=\sum^6_{k=0}(-1){6\choose k}(6-k)^n\]

which is $1-6\left(\frac{5}{6}\right)^n+15\left(\frac{4}{6}\right)^n-20\left(\frac{3}{6}\right)^n+15\left(\frac{2}{6}\right)^n-6\left(\frac{1}{6}\right)^n$.\\

And the probability is this number divided by $6^n$.\\

\textbf{(Incorrect) alternative thinking:} We have $6!$ ways to pick out $6$ out of $n$, then we don't care about the rest numbers here. Then we have

\[6!{n \choose 6}6^{n-6}\]

ways. But this has ``double-counting'' issue! Example: $123324415566$.\\

|||||||||||||||||||||||||||||||||||

\begin{table}[htbp!] 
	\centering
	\begin{tabular}{|c|c|c|c|c|c|}
		\hline
		a & a & b & b & c & d \\
		\hline
		f & g & e & e & c & d\\
		\hline
		f & g & i & j & l & l\\
		\hline
		h & h & i & j & m & m\\
		\hline
		n & o & q & q & r & r\\
		\hline
		n & o & p & p & s & s\\
		\hline
	\end{tabular}
	\caption{a $6\times 6$ chessboard filled with $2\times 1$ domino tiles} % name of the table, automatically ordered
\end{table}

\textbf{Observation 1:} at least one of horizontal or vertical lines is not cut.\\

\textbf{Observation 2:} a line is cut an even number of times.

Suppose every line is cut, $10$ lines, it we need at least $2\times 10$ dominos to cut those lines, but we only have $18$. So we are done.\\

\paragraph{A game:} Two choices, can take either Box1 or Box1 and Box2 together. Box2 has \$1,000 dollars. Box1 has a predictor (yesterday) predicted which of these two options you would choose:

\begin{itemize}
	\item If it thinks you are going to take both boxes, there's nothing in Box1.
	\item If it thinks you are going to take only Box1, there's \$1,000,000 in Box1.
\end{itemize}

\begin{table}[htbp!] 
	\centering
	\begin{tabular}{|c|c|c|}
		\hline
		 & Predictor is right & Predictor is wrong \\
		\hline
		Box 1 & 1,000,000 & 0 \\
		\hline
		Box 1\&2 & 1,000 & 1,001,000 \\
		\hline
	\end{tabular}
	\caption{A game} % name of the table, automatically ordered
\end{table}

Predictor is ``always'' right (99.99\% $\simeq$ = 100\%).

expected pay-off of just take Box1: $99.99\%\times 10^6 + .001\%\times 0=999900$

expected pay-off of taking Box1\&2: $99.99\%\times 10^3 + .001\%\times 1001000=1000$\\

\[free will\]

But the prediction was made yesterday, and we know there are money in two boxes, why don't we just take both boxes?

\end{document}

