\documentclass[a4paper, 11pt, twoside]{article}
\usepackage{amssymb}
\usepackage{amsmath}
\begin{document}
\title{STAT6046 week 3 lecture 5}
\author{Rui Qiu}

\maketitle

Recall what we learned last week, $\delta$ is the \textbf{force of interest rate.}

Suppose we have $\$1$ amount of dollar, invested. And we know after 1 year, we have an amount of $S(1)$. \textbf{There could be multiple ways/paths from $S(0)$ to $S(1)$.}

But note that the effective rate of interest is always $i=\frac{S(1)-1}{1}.$

Why not $S'(t)$? The 1st derivative of $S(t)$ is not a good measurement of growth. (example)

So we better use the force of interest:

\[\frac{S'(t)}{S(t)}=\delta.\]\\

\paragraph{1. Accumulated Value using $\delta$.}

\[
\delta_t=\frac{S'(t)}{S(t)}=\frac{d}{dt}\ln(S(t))
\]

\[S(n)=S(0)\cdot \exp(\int^n_0\delta_t dt)\]

\textbf{Proof:}

\[
\begin{split}
	\int^{t=n}_{t=0}\delta_t dt &= \int^n_0 \frac{d}{dt}\cdot \ln \bigg[S(t)\bigg] dt \\
	&= \ln[S(n)] - \ln[S(0)]\\
	S(n) &\implies S(0)\cdot \exp\bigg[\int^n_0\delta_t dt \bigg]
\end{split}
\]

And some other review stuffs.

\pagebreak

\paragraph{Annuities}

In this class we mainly focus on 3 types of annuities: \textbf{immediate annuity, annuity due and deferred annuity.}\\

\paragraph{Immediate Annuity}
\end{document}